%
% API Documentation for SpacePy: Space Science Tools for Python
% Module spacepy.seapy
%
% Generated by epydoc 3.0.1
% [Mon May 17 14:36:32 2010]
%

%%%%%%%%%%%%%%%%%%%%%%%%%%%%%%%%%%%%%%%%%%%%%%%%%%%%%%%%%%%%%%%%%%%%%%%%%%%
%%                          Module Description                           %%
%%%%%%%%%%%%%%%%%%%%%%%%%%%%%%%%%%%%%%%%%%%%%%%%%%%%%%%%%%%%%%%%%%%%%%%%%%%

    \index{spacepy \textit{(package)}!spacepy.seapy \textit{(module)}|(}
\section{Module spacepy.seapy}

    \label{spacepy:seapy}
SeaPy -- Superposed Epoch in Python.

This module contains superposed epoch class types and a variety of 
functions for using on superposed epoch objects. Each instance must be 
initialized with (assuming import seapy as se):

\begin{alltt}
\pysrcprompt{{\textgreater}{\textgreater}{\textgreater} }obj = se.Sea(data, times, epochs)\end{alltt}
To perform a superposed epoch analysis

\begin{alltt}
\pysrcprompt{{\textgreater}{\textgreater}{\textgreater} }obj.sea()\end{alltt}
To plot

\begin{alltt}
\pysrcprompt{{\textgreater}{\textgreater}{\textgreater} }obj.seplot()\end{alltt}
If multiple SeaPy objects exist, these can be combined into a single object

\begin{alltt}
\pysrcprompt{{\textgreater}{\textgreater}{\textgreater} }objdict = seadict([obj1, obj2],[\pysrcstring{'obj1name'},\pysrcstring{'obj2name'}])\end{alltt}
and then used to create a multipanel plot

\begin{alltt}
\pysrcprompt{{\textgreater}{\textgreater}{\textgreater} }seamulti(objdict)\end{alltt}
For two-dimensional superposed epoch analyses, initialize an Sea2d() 
instance

\begin{alltt}
\pysrcprompt{{\textgreater}{\textgreater}{\textgreater} }obj = se.Sea2d(data, times, epochs, y=[4., 12.])\end{alltt}
All object methods are the same as for the 1D object. Also, the seamulti() 
function should accept both 1D and 2D objects, even mixed together. 
Currently, the seplot() method is recommended for 2D SEA.

--++-- By Steve Morley --++--

smorley@lanl.gov/morley\_steve@hotmail.com, Los Alamos National Laboratory,
ISR-1, PO Box 1663, Los Alamos, NM 87545

\textbf{Author:} Steve Morley (smorley@lanl.gov/morley\_steve@hotmail.com)




%%%%%%%%%%%%%%%%%%%%%%%%%%%%%%%%%%%%%%%%%%%%%%%%%%%%%%%%%%%%%%%%%%%%%%%%%%%
%%                               Functions                               %%
%%%%%%%%%%%%%%%%%%%%%%%%%%%%%%%%%%%%%%%%%%%%%%%%%%%%%%%%%%%%%%%%%%%%%%%%%%%

  \subsection{Functions}

    \label{spacepy:seapy:seadict}
    \index{spacepy \textit{(package)}!spacepy.seapy \textit{(module)}!spacepy.seapy.seadict \textit{(function)}}

    \vspace{0.5ex}

\hspace{.8\funcindent}\begin{boxedminipage}{\funcwidth}

    \raggedright \textbf{seadict}(\textit{objlist}, \textit{namelist})

    \vspace{-1.5ex}

    \rule{\textwidth}{0.5\fboxrule}
\setlength{\parskip}{2ex}
    Function to create dictionary of SeaPy.Sea objects.

    Inputs: objlist - List of Sea objects. namelist - List of variable 
    labels for input objects. Optional keyword(s): namelist = List 
    containing names for y-axes.

\setlength{\parskip}{1ex}
    \end{boxedminipage}

    \label{spacepy:seapy:multisea}
    \index{spacepy \textit{(package)}!spacepy.seapy \textit{(module)}!spacepy.seapy.multisea \textit{(function)}}

    \vspace{0.5ex}

\hspace{.8\funcindent}\begin{boxedminipage}{\funcwidth}

    \raggedright \textbf{multisea}(\textit{dictobj}, \textit{n\_cols}={\tt 1}, \textit{epochline}={\tt False}, \textit{usrlimx}={\tt \texttt{[}\texttt{]}}, \textit{usrlimy}={\tt \texttt{[}\texttt{]}}, \textit{xunits}={\tt \texttt{'}\texttt{}\texttt{'}}, \textit{show}={\tt True}, \textit{zunits}={\tt \texttt{'}\texttt{}\texttt{'}}, \textit{zlog}={\tt True}, \textit{figsize}={\tt None}, \textit{dpi}={\tt 300})

    \vspace{-1.5ex}

    \rule{\textwidth}{0.5\fboxrule}
\setlength{\parskip}{2ex}
    Function to create multipanel plot of superposed epoch analyses.

    Inputs: Dictionary of Sea objects (from superposedepoch.seadict()). 
    Optional keyword(s): epochline (default = False) - put vertical line at
    zero epoch. usrlimy (default = []) - override automatic y-limits on 
    plot (same for all plots). show (default = True) - shows plot; set to 
    false to output plot object to variable x/zunits - Units for labelling 
    x and z axes, if required figsize - tuple of (width, height) in inches 
    dpi (default=300) - figure resolution in dots per inch n\_cols - Number
    of columns: not yet implemented.

    Output: Plot of input object median and bounds (ci, mad, quartiles - 
    see sea()). If keyword 'show' is False, output is a plot object.

\setlength{\parskip}{1ex}
    \end{boxedminipage}

    \label{spacepy:seapy:readepochs}
    \index{spacepy \textit{(package)}!spacepy.seapy \textit{(module)}!spacepy.seapy.readepochs \textit{(function)}}

    \vspace{0.5ex}

\hspace{.8\funcindent}\begin{boxedminipage}{\funcwidth}

    \raggedright \textbf{readepochs}(\textit{fname}, \textit{iso}={\tt False}, \textit{isofmt}={\tt \texttt{'}\texttt{\%Y-\%m-\%dT\%H:\%M:\%S}\texttt{'}})

    \vspace{-1.5ex}

    \rule{\textwidth}{0.5\fboxrule}
\setlength{\parskip}{2ex}
    Read epochs from text file assuming YYYY MM DD hh mm ss format

    Input: Filename (include path) Optional inputs: iso (default = False), 
    read in ISO date format isofmt (default is YYYY-mm-ddTHH:MM:SS, code is
    \%Y-\%m-\%dT\%H:\%M:\%S) Output: epochs (type=list)

\setlength{\parskip}{1ex}
    \end{boxedminipage}


%%%%%%%%%%%%%%%%%%%%%%%%%%%%%%%%%%%%%%%%%%%%%%%%%%%%%%%%%%%%%%%%%%%%%%%%%%%
%%                               Variables                               %%
%%%%%%%%%%%%%%%%%%%%%%%%%%%%%%%%%%%%%%%%%%%%%%%%%%%%%%%%%%%%%%%%%%%%%%%%%%%

  \subsection{Variables}

    \vspace{-1cm}
\hspace{\varindent}\begin{longtable}{|p{\varnamewidth}|p{\vardescrwidth}|l}
\cline{1-2}
\cline{1-2} \centering \textbf{Name} & \centering \textbf{Description}& \\
\cline{1-2}
\endhead\cline{1-2}\multicolumn{3}{r}{\small\textit{continued on next page}}\\\endfoot\cline{1-2}
\endlastfoot\raggedright \_\-\_\-p\-a\-c\-k\-a\-g\-e\-\_\-\_\- & \raggedright \textbf{Value:} 
{\tt \texttt{'}\texttt{spacepy}\texttt{'}}&\\
\cline{1-2}
\end{longtable}


%%%%%%%%%%%%%%%%%%%%%%%%%%%%%%%%%%%%%%%%%%%%%%%%%%%%%%%%%%%%%%%%%%%%%%%%%%%
%%                           Class Description                           %%
%%%%%%%%%%%%%%%%%%%%%%%%%%%%%%%%%%%%%%%%%%%%%%%%%%%%%%%%%%%%%%%%%%%%%%%%%%%

    \index{spacepy \textit{(package)}!spacepy.seapy \textit{(module)}!spacepy.seapy.Sea \textit{(class)}|(}
\subsection{Class Sea}

    \label{spacepy:seapy:Sea}
\begin{tabular}{cccccc}
% Line for object, linespec=[False]
\multicolumn{2}{r}{\settowidth{\BCL}{object}\multirow{2}{\BCL}{object}}
&&
  \\\cline{3-3}
  &&\multicolumn{1}{c|}{}
&&
  \\
&&\multicolumn{2}{l}{\textbf{spacepy.seapy.Sea}}
\end{tabular}

\textbf{Known Subclasses:} spacepy.seapy.Sea2d

SeaPy Superposed epoch analysis object

Initialize object with data, times, epochs, window (half-width) and delta 
(optional). 'times' and epochs should be in some useful format Includes 
method to perform superposed epoch analysis of input data series

Output can be nicely plotted with seaplot method, or for multiple objects 
use the seamulti function


%%%%%%%%%%%%%%%%%%%%%%%%%%%%%%%%%%%%%%%%%%%%%%%%%%%%%%%%%%%%%%%%%%%%%%%%%%%
%%                                Methods                                %%
%%%%%%%%%%%%%%%%%%%%%%%%%%%%%%%%%%%%%%%%%%%%%%%%%%%%%%%%%%%%%%%%%%%%%%%%%%%

  \subsubsection{Methods}

    \vspace{0.5ex}

\hspace{.8\funcindent}\begin{boxedminipage}{\funcwidth}

    \raggedright \textbf{\_\_init\_\_}(\textit{self}, \textit{data}, \textit{times}, \textit{epochs}, \textit{window}={\tt 3.0}, \textit{delta}={\tt 1.0})

\setlength{\parskip}{2ex}
    x.\_\_init\_\_(...) initializes x; see x.\_\_class\_\_.\_\_doc\_\_ for 
    signature

\setlength{\parskip}{1ex}
      Overrides: object.\_\_init\_\_ 	extit{(inherited documentation)}

    \end{boxedminipage}

    \vspace{0.5ex}

\hspace{.8\funcindent}\begin{boxedminipage}{\funcwidth}

    \raggedright \textbf{\_\_str\_\_}(\textit{self})

    \vspace{-1.5ex}

    \rule{\textwidth}{0.5\fboxrule}
\setlength{\parskip}{2ex}
    Define String Representation of Sea object

\setlength{\parskip}{1ex}
      Overrides: object.\_\_str\_\_

    \end{boxedminipage}

    \vspace{0.5ex}

\hspace{.8\funcindent}\begin{boxedminipage}{\funcwidth}

    \raggedright \textbf{\_\_repr\_\_}(\textit{self})

    \vspace{-1.5ex}

    \rule{\textwidth}{0.5\fboxrule}
\setlength{\parskip}{2ex}
    Define String Representation of Sea object

\setlength{\parskip}{1ex}
      Overrides: object.\_\_repr\_\_

    \end{boxedminipage}

    \label{spacepy:seapy:Sea:__len__}
    \index{spacepy \textit{(package)}!spacepy.seapy \textit{(module)}!spacepy.seapy.Sea \textit{(class)}!spacepy.seapy.Sea.\_\_len\_\_ \textit{(method)}}

    \vspace{0.5ex}

\hspace{.8\funcindent}\begin{boxedminipage}{\funcwidth}

    \raggedright \textbf{\_\_len\_\_}(\textit{self})

    \vspace{-1.5ex}

    \rule{\textwidth}{0.5\fboxrule}
\setlength{\parskip}{2ex}
    Calling len(obj) will return the number of epochs.

\setlength{\parskip}{1ex}
    \end{boxedminipage}

    \label{spacepy:seapy:Sea:restoreepochs}
    \index{spacepy \textit{(package)}!spacepy.seapy \textit{(module)}!spacepy.seapy.Sea \textit{(class)}!spacepy.seapy.Sea.restoreepochs \textit{(method)}}

    \vspace{0.5ex}

\hspace{.8\funcindent}\begin{boxedminipage}{\funcwidth}

    \raggedright \textbf{restoreepochs}(\textit{self})

    \vspace{-1.5ex}

    \rule{\textwidth}{0.5\fboxrule}
\setlength{\parskip}{2ex}
    Replaces epoch times stored in obj.badepochs in the epochs attribute

    Quite why you'd want this feature I don't know.

\setlength{\parskip}{1ex}
    \end{boxedminipage}

    \label{spacepy:seapy:Sea:sea}
    \index{spacepy \textit{(package)}!spacepy.seapy \textit{(module)}!spacepy.seapy.Sea \textit{(class)}!spacepy.seapy.Sea.sea \textit{(method)}}

    \vspace{0.5ex}

\hspace{.8\funcindent}\begin{boxedminipage}{\funcwidth}

    \raggedright \textbf{sea}(\textit{self}, \textit{storedata}={\tt False}, \textit{quartiles}={\tt True}, \textit{ci}={\tt False}, \textit{mad}={\tt False}, \textit{ci\_quan}={\tt \texttt{'}\texttt{median}\texttt{'}})

    \vspace{-1.5ex}

    \rule{\textwidth}{0.5\fboxrule}
\setlength{\parskip}{2ex}
    Method called to perform superposed epoch analysis on data in object.

    Inputs: Uses object attributes obj.data, obj.times, obj.epochs, 
    obj.delta, obj.window, all of which must be available on instatiation. 
    Optional keyword(s): storedata (default = False) - saves matrix of 
    epoch windows as obj.datacube quartiles calculates the quartiles as the
    upper and lower bounds (and is default); ci will find the bootstrapped 
    confidence intervals (and requires ci\_quan to be set); mad will use 
    +/- the median absolute deviation for the bounds; ci\_quan can be set 
    to 'median' or 'mean'

    A basic plot can be raised with the obj.seplot() method

\setlength{\parskip}{1ex}
    \end{boxedminipage}

    \label{spacepy:seapy:Sea:plot}
    \index{spacepy \textit{(package)}!spacepy.seapy \textit{(module)}!spacepy.seapy.Sea \textit{(class)}!spacepy.seapy.Sea.plot \textit{(method)}}

    \vspace{0.5ex}

\hspace{.8\funcindent}\begin{boxedminipage}{\funcwidth}

    \raggedright \textbf{plot}(\textit{self}, \textit{xquan}={\tt \texttt{'}\texttt{Time Since Epoch}\texttt{'}}, \textit{yquan}={\tt \texttt{'}\texttt{}\texttt{'}}, \textit{xunits}={\tt \texttt{'}\texttt{}\texttt{'}}, \textit{yunits}={\tt \texttt{'}\texttt{}\texttt{'}}, \textit{epochline}={\tt False}, \textit{usrlimy}={\tt \texttt{[}\texttt{]}}, \textit{figsize}={\tt None}, \textit{dpi}={\tt 300})

    \vspace{-1.5ex}

    \rule{\textwidth}{0.5\fboxrule}
\setlength{\parskip}{2ex}
    Method called to create basic plot of superposed epoch analysis.

    Inputs: Uses object attributes created by the obj.sea() method.

    Optional keyword(s): x(y)quan (default = 'Time since epoch' (None)) - 
    x(y)-axis label. x(y)units (default = None (None)) - x(y)-axis units. 
    epochline (default = False) - put vertical line at zero epoch. usrlimy 
    (default = []) - override automatic y-limits on plot.

    If both ?quan and ?units are supplied, axis label will read 'Quantity 
    Entered By User [Units]'

\setlength{\parskip}{1ex}
    \end{boxedminipage}

    \label{spacepy:seapy:Sea:plot}
    \index{spacepy \textit{(package)}!spacepy.seapy \textit{(module)}!spacepy.seapy.Sea \textit{(class)}!spacepy.seapy.Sea.plot \textit{(method)}}

    \vspace{0.5ex}

\hspace{.8\funcindent}\begin{boxedminipage}{\funcwidth}

    \raggedright \textbf{seplot}(\textit{self}, \textit{xquan}={\tt \texttt{'}\texttt{Time Since Epoch}\texttt{'}}, \textit{yquan}={\tt \texttt{'}\texttt{}\texttt{'}}, \textit{xunits}={\tt \texttt{'}\texttt{}\texttt{'}}, \textit{yunits}={\tt \texttt{'}\texttt{}\texttt{'}}, \textit{epochline}={\tt False}, \textit{usrlimy}={\tt \texttt{[}\texttt{]}}, \textit{figsize}={\tt None}, \textit{dpi}={\tt 300})

    \vspace{-1.5ex}

    \rule{\textwidth}{0.5\fboxrule}
\setlength{\parskip}{2ex}
    Method called to create basic plot of superposed epoch analysis.

    Inputs: Uses object attributes created by the obj.sea() method.

    Optional keyword(s): x(y)quan (default = 'Time since epoch' (None)) - 
    x(y)-axis label. x(y)units (default = None (None)) - x(y)-axis units. 
    epochline (default = False) - put vertical line at zero epoch. usrlimy 
    (default = []) - override automatic y-limits on plot.

    If both ?quan and ?units are supplied, axis label will read 'Quantity 
    Entered By User [Units]'

\setlength{\parskip}{1ex}
    \end{boxedminipage}


\large{\textbf{\textit{Inherited from object}}}

\begin{quote}
\_\_delattr\_\_(), \_\_format\_\_(), \_\_getattribute\_\_(), \_\_hash\_\_(), \_\_new\_\_(), \_\_reduce\_\_(), \_\_reduce\_ex\_\_(), \_\_setattr\_\_(), \_\_sizeof\_\_(), \_\_subclasshook\_\_()
\end{quote}

%%%%%%%%%%%%%%%%%%%%%%%%%%%%%%%%%%%%%%%%%%%%%%%%%%%%%%%%%%%%%%%%%%%%%%%%%%%
%%                              Properties                               %%
%%%%%%%%%%%%%%%%%%%%%%%%%%%%%%%%%%%%%%%%%%%%%%%%%%%%%%%%%%%%%%%%%%%%%%%%%%%

  \subsubsection{Properties}

    \vspace{-1cm}
\hspace{\varindent}\begin{longtable}{|p{\varnamewidth}|p{\vardescrwidth}|l}
\cline{1-2}
\cline{1-2} \centering \textbf{Name} & \centering \textbf{Description}& \\
\cline{1-2}
\endhead\cline{1-2}\multicolumn{3}{r}{\small\textit{continued on next page}}\\\endfoot\cline{1-2}
\endlastfoot\multicolumn{2}{|l|}{\textit{Inherited from object}}\\
\multicolumn{2}{|p{\varwidth}|}{\raggedright \_\_class\_\_}\\
\cline{1-2}
\end{longtable}

    \index{spacepy \textit{(package)}!spacepy.seapy \textit{(module)}!spacepy.seapy.Sea \textit{(class)}|)}

%%%%%%%%%%%%%%%%%%%%%%%%%%%%%%%%%%%%%%%%%%%%%%%%%%%%%%%%%%%%%%%%%%%%%%%%%%%
%%                           Class Description                           %%
%%%%%%%%%%%%%%%%%%%%%%%%%%%%%%%%%%%%%%%%%%%%%%%%%%%%%%%%%%%%%%%%%%%%%%%%%%%

    \index{spacepy \textit{(package)}!spacepy.seapy \textit{(module)}!spacepy.seapy.Sea2d \textit{(class)}|(}
\subsection{Class Sea2d}

    \label{spacepy:seapy:Sea2d}
\begin{tabular}{cccccccc}
% Line for object, linespec=[False, False]
\multicolumn{2}{r}{\settowidth{\BCL}{object}\multirow{2}{\BCL}{object}}
&&
&&
  \\\cline{3-3}
  &&\multicolumn{1}{c|}{}
&&
&&
  \\
% Line for spacepy.seapy.Sea, linespec=[False]
\multicolumn{4}{r}{\settowidth{\BCL}{spacepy.seapy.Sea}\multirow{2}{\BCL}{spacepy.seapy.Sea}}
&&
  \\\cline{5-5}
  &&&&\multicolumn{1}{c|}{}
&&
  \\
&&&&\multicolumn{2}{l}{\textbf{spacepy.seapy.Sea2d}}
\end{tabular}

SeaPy 2D Superposed epoch analysis object

Initialize object with data, times, epochs, window (half-width), delta 
(optional), and y (two-element vector with max and min of y;optional) 
'times' and epochs should be in some useful format Includes method to 
perform superposed epoch analysis of input data series

Output can be nicely plotted with seplot method, or for multiple objects 
use the seamulti function


%%%%%%%%%%%%%%%%%%%%%%%%%%%%%%%%%%%%%%%%%%%%%%%%%%%%%%%%%%%%%%%%%%%%%%%%%%%
%%                                Methods                                %%
%%%%%%%%%%%%%%%%%%%%%%%%%%%%%%%%%%%%%%%%%%%%%%%%%%%%%%%%%%%%%%%%%%%%%%%%%%%

  \subsubsection{Methods}

    \vspace{0.5ex}

\hspace{.8\funcindent}\begin{boxedminipage}{\funcwidth}

    \raggedright \textbf{\_\_init\_\_}(\textit{self}, \textit{data}, \textit{times}, \textit{epochs}, \textit{window}={\tt 3.0}, \textit{delta}={\tt 1.0}, \textit{y}={\tt \texttt{[}\texttt{]}})

\setlength{\parskip}{2ex}
    x.\_\_init\_\_(...) initializes x; see x.\_\_class\_\_.\_\_doc\_\_ for 
    signature

\setlength{\parskip}{1ex}
      Overrides: object.\_\_init\_\_ 	extit{(inherited documentation)}

    \end{boxedminipage}

    \vspace{0.5ex}

\hspace{.8\funcindent}\begin{boxedminipage}{\funcwidth}

    \raggedright \textbf{sea}(\textit{self}, \textit{storedata}={\tt False}, \textit{quartiles}={\tt True}, \textit{ci}={\tt False}, \textit{mad}={\tt False}, \textit{ci\_quan}={\tt \texttt{'}\texttt{median}\texttt{'}}, \textit{nmask}={\tt 1})

    \vspace{-1.5ex}

    \rule{\textwidth}{0.5\fboxrule}
\setlength{\parskip}{2ex}
    Method called to perform 2D superposed epoch analysis on data in 
    object.

    Inputs: Uses object attributes obj.data, obj.times, obj.epochs, 
    obj.delta, obj.window, all of which must be available on instatiation. 
    Optional keyword(s): storedata (default = False) - saves matrix of 
    epoch windows as obj.datacube quartiles calculates the interquartile 
    range to show the spread (and is default); ci will find the 
    bootstrapped confidence interval (and requires ci\_quan to be set); mad
    will use the median absolute deviation for the spread; ci\_quan can be 
    set to 'median' or 'mean'

    A basic plot can be raised with the obj.seplot() method

\setlength{\parskip}{1ex}
      Overrides: spacepy.seapy.Sea.sea

    \end{boxedminipage}

    \vspace{0.5ex}

\hspace{.8\funcindent}\begin{boxedminipage}{\funcwidth}

    \raggedright \textbf{plot}(\textit{self}, \textit{xquan}={\tt \texttt{'}\texttt{Time Since Epoch}\texttt{'}}, \textit{yquan}={\tt \texttt{'}\texttt{}\texttt{'}}, \textit{xunits}={\tt \texttt{'}\texttt{}\texttt{'}}, \textit{yunits}={\tt \texttt{'}\texttt{}\texttt{'}}, \textit{zunits}={\tt \texttt{'}\texttt{}\texttt{'}}, \textit{epochline}={\tt False}, \textit{usrlimy}={\tt \texttt{[}\texttt{]}}, \textit{show}={\tt True}, \textit{zlog}={\tt True}, \textit{figsize}={\tt None}, \textit{dpi}={\tt 300})

    \vspace{-1.5ex}

    \rule{\textwidth}{0.5\fboxrule}
\setlength{\parskip}{2ex}
    Method called to create basic plot of 2D superposed epoch analysis.

    Inputs: Uses object attributes created by the obj.sea() method. 
    Optional keyword(s): x(y)quan (default = 'Time since epoch' (None)) - 
    x(y)-axis label. x(y/z)units (default = None (None)) - x(y/z)-axis 
    units. epochline (default = False) - put vertical line at zero epoch. 
    usrlimy (default = []) - override automatic y-limits on plot. show 
    (default = True) - shows plot; set to false to output plot object to 
    variable figsize - tuple of (width, height) in inches dpi (default=300)
    - figure resolution in dots per inch If both ?quan and ?units are 
    supplied, axis label will read 'Quantity Entered By User [Units]'

\setlength{\parskip}{1ex}
      Overrides: spacepy.seapy.Sea.plot

    \end{boxedminipage}

    \vspace{0.5ex}

\hspace{.8\funcindent}\begin{boxedminipage}{\funcwidth}

    \raggedright \textbf{seplot}(\textit{self}, \textit{xquan}={\tt \texttt{'}\texttt{Time Since Epoch}\texttt{'}}, \textit{yquan}={\tt \texttt{'}\texttt{}\texttt{'}}, \textit{xunits}={\tt \texttt{'}\texttt{}\texttt{'}}, \textit{yunits}={\tt \texttt{'}\texttt{}\texttt{'}}, \textit{zunits}={\tt \texttt{'}\texttt{}\texttt{'}}, \textit{epochline}={\tt False}, \textit{usrlimy}={\tt \texttt{[}\texttt{]}}, \textit{show}={\tt True}, \textit{zlog}={\tt True}, \textit{figsize}={\tt None}, \textit{dpi}={\tt 300})

    \vspace{-1.5ex}

    \rule{\textwidth}{0.5\fboxrule}
\setlength{\parskip}{2ex}
    Method called to create basic plot of 2D superposed epoch analysis.

    Inputs: Uses object attributes created by the obj.sea() method. 
    Optional keyword(s): x(y)quan (default = 'Time since epoch' (None)) - 
    x(y)-axis label. x(y/z)units (default = None (None)) - x(y/z)-axis 
    units. epochline (default = False) - put vertical line at zero epoch. 
    usrlimy (default = []) - override automatic y-limits on plot. show 
    (default = True) - shows plot; set to false to output plot object to 
    variable figsize - tuple of (width, height) in inches dpi (default=300)
    - figure resolution in dots per inch If both ?quan and ?units are 
    supplied, axis label will read 'Quantity Entered By User [Units]'

\setlength{\parskip}{1ex}
      Overrides: spacepy.seapy.Sea.seplot

    \end{boxedminipage}


\large{\textbf{\textit{Inherited from spacepy.seapy.Sea\textit{(Section \ref{spacepy:seapy:Sea})}}}}

\begin{quote}
\_\_len\_\_(), \_\_repr\_\_(), \_\_str\_\_(), restoreepochs()
\end{quote}

\large{\textbf{\textit{Inherited from object}}}

\begin{quote}
\_\_delattr\_\_(), \_\_format\_\_(), \_\_getattribute\_\_(), \_\_hash\_\_(), \_\_new\_\_(), \_\_reduce\_\_(), \_\_reduce\_ex\_\_(), \_\_setattr\_\_(), \_\_sizeof\_\_(), \_\_subclasshook\_\_()
\end{quote}

%%%%%%%%%%%%%%%%%%%%%%%%%%%%%%%%%%%%%%%%%%%%%%%%%%%%%%%%%%%%%%%%%%%%%%%%%%%
%%                              Properties                               %%
%%%%%%%%%%%%%%%%%%%%%%%%%%%%%%%%%%%%%%%%%%%%%%%%%%%%%%%%%%%%%%%%%%%%%%%%%%%

  \subsubsection{Properties}

    \vspace{-1cm}
\hspace{\varindent}\begin{longtable}{|p{\varnamewidth}|p{\vardescrwidth}|l}
\cline{1-2}
\cline{1-2} \centering \textbf{Name} & \centering \textbf{Description}& \\
\cline{1-2}
\endhead\cline{1-2}\multicolumn{3}{r}{\small\textit{continued on next page}}\\\endfoot\cline{1-2}
\endlastfoot\multicolumn{2}{|l|}{\textit{Inherited from object}}\\
\multicolumn{2}{|p{\varwidth}|}{\raggedright \_\_class\_\_}\\
\cline{1-2}
\end{longtable}

    \index{spacepy \textit{(package)}!spacepy.seapy \textit{(module)}!spacepy.seapy.Sea2d \textit{(class)}|)}
    \index{spacepy \textit{(package)}!spacepy.seapy \textit{(module)}|)}

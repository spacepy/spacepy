%
% API Documentation for SpacePy: Space Science Tools for Python
% Module spacepy.ticktock
%
% Generated by epydoc 3.0.1
% [Tue Feb  2 17:36:17 2010]
%

%%%%%%%%%%%%%%%%%%%%%%%%%%%%%%%%%%%%%%%%%%%%%%%%%%%%%%%%%%%%%%%%%%%%%%%%%%%
%%                          Module Description                           %%
%%%%%%%%%%%%%%%%%%%%%%%%%%%%%%%%%%%%%%%%%%%%%%%%%%%%%%%%%%%%%%%%%%%%%%%%%%%

    \index{spacepy \textit{(package)}!spacepy.ticktock \textit{(module)}|(}
\section{Module spacepy.ticktock}

    \label{spacepy:ticktock}
toolbox with time functions

\textbf{Version:} 0.01



\textbf{Author:} J. Koller




%%%%%%%%%%%%%%%%%%%%%%%%%%%%%%%%%%%%%%%%%%%%%%%%%%%%%%%%%%%%%%%%%%%%%%%%%%%
%%                               Functions                               %%
%%%%%%%%%%%%%%%%%%%%%%%%%%%%%%%%%%%%%%%%%%%%%%%%%%%%%%%%%%%%%%%%%%%%%%%%%%%

  \subsection{Functions}

    \label{spacepy:ticktock:getCDF}
    \index{spacepy \textit{(package)}!spacepy.ticktock \textit{(module)}!spacepy.ticktock.getCDF \textit{(function)}}

    \vspace{0.5ex}

\hspace{.8\funcindent}\begin{boxedminipage}{\funcwidth}

    \raggedright \textbf{getCDF}(\textit{data}, \textit{dtype})

    \vspace{-1.5ex}

    \rule{\textwidth}{0.5\fboxrule}
\setlength{\parskip}{2ex}
    Return CDF time which is milliseconds since 01-Jan-0000 00:00:00.000. 
    Year zero" is a convention chosen by NSSDC to measure epoch values. 
    This date is more commonly referred to as 1 BC. Remember that 1 BC was 
    a leap year. The CDF date/time calculations do not take into account 
    the changes to the Gregorian calendar, and cannot be directly converted
    into Julian date/times.

    Possible data types: ISO: ISO standard format like 
    '2002-02-25T12:20:30' UTC: datetime object with UTC time TAI: elapsed 
    seconds since 1958/1/1 (includes leap seconds) UNX: elapsed seconds 
    since 1970/1/1 (all days have 86400 secs of occasionally unequal 
    lenghts) JD: Julian days elapsed MJD: Modified Julian days RTD: Rata 
    Die days elapsed since 1/1/1 CDF: CDF epoch time: milliseconds since 
    1/0/0 00:00:00.000

    (section) Input:

      \begin{itemize}
      \setlength{\parskip}{0.6ex}
        \item data (int, datetime, float, string as singles or array) : time 
          stamp

        \item dtype (string) : data type for data ISO, UTC, TAI, UNX, JD, MJD, 
          RTD

      \end{itemize}

    (section) Returns:

      \begin{itemize}
      \setlength{\parskip}{0.6ex}
        \item CDF (float as single or array) : days elapsed since Jan. 1st

      \end{itemize}

    (section) Example:

\begin{alltt}
\pysrcprompt{{\textgreater}{\textgreater}{\textgreater} }x=datetime.datetime(2002,2,25,12,20,30)
\pysrcprompt{{\textgreater}{\textgreater}{\textgreater} }getCDF(x, \pysrcstring{'UTC'})\end{alltt}
    (section) Author:

      Josef Koller, Los Alamos National Lab (jkoller@lanl.gov)

    (section) Version:

      V1: 02-Feb-2010

\setlength{\parskip}{1ex}
    \end{boxedminipage}

    \label{spacepy:ticktock:getDOY}
    \index{spacepy \textit{(package)}!spacepy.ticktock \textit{(module)}!spacepy.ticktock.getDOY \textit{(function)}}

    \vspace{0.5ex}

\hspace{.8\funcindent}\begin{boxedminipage}{\funcwidth}

    \raggedright \textbf{getDOY}(\textit{data}, \textit{dtype})

    \vspace{-1.5ex}

    \rule{\textwidth}{0.5\fboxrule}
\setlength{\parskip}{2ex}
    extract DOY (days since January 1st of given year)

    Possible data types: ISO: ISO standard format like 
    '2002-02-25T12:20:30' UTC: datetime object with UTC time TAI: elapsed 
    seconds since 1958/1/1 (includes leap seconds) UNX: elapsed seconds 
    since 1970/1/1 (all days have 86400 secs of occasionally unequal 
    lenghts) JD: Julian days elapsed MJD: Modified Julian days RTD: Rata 
    Die days elapsed since 1/1/1

    (section) Input:

      \begin{itemize}
      \setlength{\parskip}{0.6ex}
        \item data (int, datetime, float, string as singles or array) : time 
          stamp

        \item dtype (string) : data type for data ISO, UTC, TAI, UNX, JD, MJD, 
          RTD

      \end{itemize}

    (section) Returns:

      \begin{itemize}
      \setlength{\parskip}{0.6ex}
        \item DOY (float as single or array) : days elapsed since Jan. 1st

      \end{itemize}

    (section) Example:

\begin{alltt}
\pysrcprompt{{\textgreater}{\textgreater}{\textgreater} }x=datetime.datetime(2002,2,25,12,20,30)
\pysrcprompt{{\textgreater}{\textgreater}{\textgreater} }getDOY(x, \pysrcstring{'UTC'})\end{alltt}
    (section) Author:

      Josef Koller, Los Alamos National Lab (jkoller@lanl.gov)

    (section) Version:

      V1: 25-Jan-2010

\setlength{\parskip}{1ex}
    \end{boxedminipage}

    \label{spacepy:ticktock:getJD}
    \index{spacepy \textit{(package)}!spacepy.ticktock \textit{(module)}!spacepy.ticktock.getJD \textit{(function)}}

    \vspace{0.5ex}

\hspace{.8\funcindent}\begin{boxedminipage}{\funcwidth}

    \raggedright \textbf{getJD}(\textit{data}, \textit{dtype})

    \vspace{-1.5ex}

    \rule{\textwidth}{0.5\fboxrule}
\setlength{\parskip}{2ex}
    convert dtype data into Julian Date (JD)

    Possible data types: ISO: ISO standard format like 
    '2002-02-25T12:20:30' UTC: datetime object with UTC time TAI: elapsed 
    seconds since 1958/1/1 (includes leap seconds) UNX: elapsed seconds 
    since 1970/1/1 (all days have 86400 secs of occasionally unequal 
    lenghts) JD: Julian days elapsed MJD: Modified Julian days RTD: Rata 
    Die days elapsed since 1/1/1

    (section) Input:

      \begin{itemize}
      \setlength{\parskip}{0.6ex}
        \item data (int, datetime, float, string as singles or array) : time 
          stamp

        \item dtype (string) : data type for data ISO, UTC, TAI, UNX, JD, MJD, 
          RTD

      \end{itemize}

    (section) Returns:

      \begin{itemize}
      \setlength{\parskip}{0.6ex}
        \item JD (float as single or array) : elapsed days since 12:00 January 
          1, 4713 BC

      \end{itemize}

    (section) Example:

\begin{alltt}
\pysrcprompt{{\textgreater}{\textgreater}{\textgreater} }x=datetime.datetime(2002,2,25,12,20,30)
\pysrcprompt{{\textgreater}{\textgreater}{\textgreater} }getJD(x, \pysrcstring{'UTC'})
\pysrcoutput{2452331.0142361112}\end{alltt}
    (section) Author:

      Josef Koller, Los Alamos National Lab (jkoller@lanl.gov)

    (section) Version:

      V1: 20-Jan-2010 V2: 25-Jan-2010: added array support (JK)

\setlength{\parskip}{1ex}
    \end{boxedminipage}

    \label{spacepy:ticktock:getMJD}
    \index{spacepy \textit{(package)}!spacepy.ticktock \textit{(module)}!spacepy.ticktock.getMJD \textit{(function)}}

    \vspace{0.5ex}

\hspace{.8\funcindent}\begin{boxedminipage}{\funcwidth}

    \raggedright \textbf{getMJD}(\textit{data}, \textit{dtype})

    \vspace{-1.5ex}

    \rule{\textwidth}{0.5\fboxrule}
\setlength{\parskip}{2ex}
    convert dtype data into MJD (modified Julian date)

    Possible data types: ISO: ISO standard format like 
    '2002-02-25T12:20:30' UTC: datetime object with UTC time TAI: elapsed 
    seconds since 1958/1/1 (includes leap seconds) UNX: elapsed seconds 
    since 1970/1/1 (all days have 86400 secs of occasionally unequal 
    lenghts) JD: Julian days elapsed MJD: Modified Julian days RTD: Rata 
    Die days elapsed since 1/1/1

    (section) Input:

      \begin{itemize}
      \setlength{\parskip}{0.6ex}
        \item data (int, datetime, float, string as singles or array) : time 
          stamp

        \item dtype (string) : data type for data ISO, UTC, TAI, UNX, JD, MJD, 
          RTD

      \end{itemize}

    (section) Returns:

      \begin{itemize}
      \setlength{\parskip}{0.6ex}
        \item MJD (float as single or array) : elapsed days since November 17, 
          1858 (Julian date was 2,400 000)

      \end{itemize}

    (section) Example:

\begin{alltt}
\pysrcprompt{{\textgreater}{\textgreater}{\textgreater} }x=datetime.datetime(2002,2,25,12,20,30)
\pysrcprompt{{\textgreater}{\textgreater}{\textgreater} }getMJD(x, \pysrcstring{'UTC'})
\pysrcoutput{52330.514236111194}\end{alltt}
    (section) Author:

      Josef Koller, Los Alamos National Lab (jkoller@lanl.gov)

    (section) Version:

      V1: 20-Jan-2010 V2: 25-Jan-2010: added support for arrays (JK)

\setlength{\parskip}{1ex}
    \end{boxedminipage}

    \label{spacepy:ticktock:getUNX}
    \index{spacepy \textit{(package)}!spacepy.ticktock \textit{(module)}!spacepy.ticktock.getUNX \textit{(function)}}

    \vspace{0.5ex}

\hspace{.8\funcindent}\begin{boxedminipage}{\funcwidth}

    \raggedright \textbf{getUNX}(\textit{data}, \textit{dtype})

    \vspace{-1.5ex}

    \rule{\textwidth}{0.5\fboxrule}
\setlength{\parskip}{2ex}
    convert dtype data into Unix Time (Posix Time) seconds since 1970-Jan-1
    (not counting leap seconds)

    Possible data types: ISO: ISO standard format like 
    '2002-02-25T12:20:30' UTC: datetime object with UTC time TAI: elapsed 
    seconds since 1958/1/1 (includes leap seconds) UNX: elapsed seconds 
    since 1970/1/1 (all days have 86400 secs of occasionally unequal 
    lenghts) JD: Julian days elapsed MJD: Modified Julian days RTD: Rata 
    Die days elapsed since 1/1/1

    (section) Input:

      \begin{itemize}
      \setlength{\parskip}{0.6ex}
        \item data (int, datetime, float, string as single or array) : time 
          stamp

        \item dtype (string) : data type for data ISO, UTC, TAI, UNX, JD, MJD, 
          RTD

      \end{itemize}

    (section) Returns:

      \begin{itemize}
      \setlength{\parskip}{0.6ex}
        \item UNX (int as single or array) : elapsed seconds since 1970/1/1 
          (not counting leap seconds)

      \end{itemize}

    (section) Example:

\begin{alltt}
\pysrcprompt{{\textgreater}{\textgreater}{\textgreater} }x=datetime.datetime(2002,2,25,12,20,30)
\pysrcprompt{{\textgreater}{\textgreater}{\textgreater} }getUNX(x, \pysrcstring{'UTC'})
\pysrcoutput{1014639630}\end{alltt}
    (section) Author:

      Josef Koller, Los Alamos National Lab (jkoller@lanl.gov)

    (section) Version:

      V1: 20-Jan-2010 V2: 25-Jan-2010: added array support (JK)

\setlength{\parskip}{1ex}
    \end{boxedminipage}

    \label{spacepy:ticktock:getRTD}
    \index{spacepy \textit{(package)}!spacepy.ticktock \textit{(module)}!spacepy.ticktock.getRTD \textit{(function)}}

    \vspace{0.5ex}

\hspace{.8\funcindent}\begin{boxedminipage}{\funcwidth}

    \raggedright \textbf{getRTD}(\textit{data}, \textit{dtype})

    \vspace{-1.5ex}

    \rule{\textwidth}{0.5\fboxrule}
\setlength{\parskip}{2ex}
    convert dtype data into Rata Die (lat.) Time (days since 1/1/0001)

    Possible data types: ISO: ISO standard format like 
    '2002-02-25T12:20:30' UTC: datetime object with UTC time TAI: elapsed 
    seconds since 1958/1/1 (includes leap seconds) UNX: elapsed seconds 
    since 1970/1/1 (all days have 86400 secs of occasionally unequal 
    lenghts) JD: Julian days elapsed MJD: Modified Julian days RTD: Rata 
    Die days elapsed since 1/1/1

    (section) Input:

      \begin{itemize}
      \setlength{\parskip}{0.6ex}
        \item data (int, datetime, float, string as singles or array) : time 
          stamp

        \item dtype (string) : data type for data ISO, UTC, TAI, UNX, JD, MJD, 
          RTD

      \end{itemize}

    (section) Returns:

      \begin{itemize}
      \setlength{\parskip}{0.6ex}
        \item RTD (float as single or array) : elapsed days since 1/1/1

      \end{itemize}

    (section) Example:

\begin{alltt}
\pysrcprompt{{\textgreater}{\textgreater}{\textgreater} }x=datetime.datetime(2002,2,25,12,20,30)
\pysrcprompt{{\textgreater}{\textgreater}{\textgreater} }getRTD(x, \pysrcstring{'UTC'})
\pysrcoutput{730906.51424768521}\end{alltt}
    (section) Author:

      Josef Koller, Los Alamos National Lab (jkoller@lanl.gov)

    (section) Version:

      V1: 20-Jan-2010 V2: 25-Jan-2010: added array support (JK)

\setlength{\parskip}{1ex}
    \end{boxedminipage}

    \label{spacepy:ticktock:getUTC}
    \index{spacepy \textit{(package)}!spacepy.ticktock \textit{(module)}!spacepy.ticktock.getUTC \textit{(function)}}

    \vspace{0.5ex}

\hspace{.8\funcindent}\begin{boxedminipage}{\funcwidth}

    \raggedright \textbf{getUTC}(\textit{data}, \textit{dtype})

    \vspace{-1.5ex}

    \rule{\textwidth}{0.5\fboxrule}
\setlength{\parskip}{2ex}
    convert dtype data into UTC object a la datetime()

    Possible data types: ISO: ISO standard format like 
    '2002-02-25T12:20:30' UTC: datetime object with UTC time TAI: elapsed 
    seconds since 1958/1/1 (includes leap seconds) UNX: elapsed seconds 
    since 1970/1/1 (all days have 86400 secs of occasionally unequal 
    lenghts) JD: Julian days elapsed MJD: Modified Julian days RTD: Rata 
    Die days elapsed since 1/1/1

    (section) Input:

      \begin{itemize}
      \setlength{\parskip}{0.6ex}
        \item data (int, datetime, float, string as singles or array) : time 
          stamp

        \item dtype (string) : data type for data ISO, UTC, TAI, UNX, JD, MJD, 
          RTD

      \end{itemize}

    (section) Returns:

      \begin{itemize}
      \setlength{\parskip}{0.6ex}
        \item UTC (datetime object as single or array) : datetime object in UTC
          time

      \end{itemize}

    (section) Example:

\begin{alltt}
\pysrcprompt{{\textgreater}{\textgreater}{\textgreater} }getUTC(1393330862, \pysrcstring{'TAI'})
\pysrcoutput{datetime.datetime(2002, 2, 25, 12, 20, 30)}\end{alltt}
    (section) Author:

      Josef Koller, Los Alamos National Lab (jkoller@lanl.gov)

    (section) Version:

      V1: 20-Jan-2010 V2: 25-Jan-2010: added array support (JK)

\setlength{\parskip}{1ex}
    \end{boxedminipage}

    \label{spacepy:ticktock:getTAI}
    \index{spacepy \textit{(package)}!spacepy.ticktock \textit{(module)}!spacepy.ticktock.getTAI \textit{(function)}}

    \vspace{0.5ex}

\hspace{.8\funcindent}\begin{boxedminipage}{\funcwidth}

    \raggedright \textbf{getTAI}(\textit{data}, \textit{dtype})

    \vspace{-1.5ex}

    \rule{\textwidth}{0.5\fboxrule}
\setlength{\parskip}{2ex}
    convert dtype data into TAI[s] number

    Possible data types: ISO: ISO standard format like 
    '2002-02-25T12:20:30' UTC: datetime object with UTC time TAI: elapsed 
    seconds since 1958/1/1 (includes leap seconds) UNX: elapsed seconds 
    since 1970/1/1 (all days have 86400 secs of occasionally unequal 
    lenghts) JD: Julian days elapsed MJD: Modified Julian days RTD: Rata 
    Die days elapsed since 1/1/1

    (section) Input:

      \begin{itemize}
      \setlength{\parskip}{0.6ex}
        \item data (int, datetime, float, string as singles or array) : time 
          stamp

        \item dtype (string) : data type for data ISO, UTC, TAI, UNX, JD, MJD, 
          RTD

      \end{itemize}

    (section) Returns:

      \begin{itemize}
      \setlength{\parskip}{0.6ex}
        \item TAI (int as single or array) : elapsed seconds since 1958/1/1 
          (includes leap seconds, i.e. all secs have equal lengths)

      \end{itemize}

    (section) Example:

\begin{alltt}
\pysrcprompt{{\textgreater}{\textgreater}{\textgreater} }x=datetime.datetime(2002,2,25,12,20,30)
\pysrcprompt{{\textgreater}{\textgreater}{\textgreater} }getTAI(x, \pysrcstring{'UTC'})
\pysrcoutput{1393330862}\end{alltt}
    (section) Author:

      Josef Koller, Los Alamos National Lab (jkoller@lanl.gov)

    (section) Version:

      V1: 20-Jan-2010 V2: 25-Jan-2010: include array support (JK)

\setlength{\parskip}{1ex}
    \end{boxedminipage}

    \label{spacepy:ticktock:getISO}
    \index{spacepy \textit{(package)}!spacepy.ticktock \textit{(module)}!spacepy.ticktock.getISO \textit{(function)}}

    \vspace{0.5ex}

\hspace{.8\funcindent}\begin{boxedminipage}{\funcwidth}

    \raggedright \textbf{getISO}(\textit{data}, \textit{dtype})

    \vspace{-1.5ex}

    \rule{\textwidth}{0.5\fboxrule}
\setlength{\parskip}{2ex}
    convert dtype data into ISO string Possible data types: ISO: ISO 
    standard format like '2002-02-25T12:20:30' UTC: datetime object with 
    UTC time TAI: elapsed seconds since 1958/1/1 (includes leap seconds) 
    UNX: elapsed seconds since 1970/1/1 (all days have 86400 secs of 
    occasionally unequal lenghts) JD: Julian days elapsed MJD: Modified 
    Julian days RTD: Rata Die days elapsed since 1/1/1

    (section) Input:

      \begin{itemize}
      \setlength{\parskip}{0.6ex}
        \item data (int, datetime, float, string as singles or arrays) : time 
          stamp

        \item dtype (string) : data type for data ISO, UTC, TAI, UNX, JD, MJD, 
          RTD

      \end{itemize}

    (section) Returns:

      \begin{itemize}
      \setlength{\parskip}{0.6ex}
        \item ISO (string or array of strings) : date in ISO format

      \end{itemize}

    (section) Example:

\begin{alltt}
\pysrcprompt{{\textgreater}{\textgreater}{\textgreater} }x=datetime.datetime(2002,2,25,12,20,30)
\pysrcprompt{{\textgreater}{\textgreater}{\textgreater} }getISO(x, \pysrcstring{'UTC'})
\pysrcoutput{'2002-02-25T12:20:30'}\end{alltt}
    (section) Author:

      Josef Koller, Los Alamos National Lab (jkoller@lanl.gov)

    (section) Version:

      V1: 20-Jan-2010 V2: 25-Jan-2010: included arary support (JK)

\setlength{\parskip}{1ex}
    \end{boxedminipage}

    \label{spacepy:ticktock:getleapsecs}
    \index{spacepy \textit{(package)}!spacepy.ticktock \textit{(module)}!spacepy.ticktock.getleapsecs \textit{(function)}}

    \vspace{0.5ex}

\hspace{.8\funcindent}\begin{boxedminipage}{\funcwidth}

    \raggedright \textbf{getleapsecs}(\textit{tup})

    \vspace{-1.5ex}

    \rule{\textwidth}{0.5\fboxrule}
\setlength{\parskip}{2ex}
    retrieve leapseconds from lookup table

    (section) Input:

      \begin{itemize}
      \setlength{\parskip}{0.6ex}
        \item tup (datetime tuple or array of tuples) : time stamp

      \end{itemize}

    (section) Returns:

      \begin{itemize}
      \setlength{\parskip}{0.6ex}
        \item secs (int or array of tuples) : leap seconds

      \end{itemize}

    (section) Example:

\begin{alltt}
\pysrcprompt{{\textgreater}{\textgreater}{\textgreater} }x=datetime.datetime(2002,2,25,12,20,30)
\pysrcprompt{{\textgreater}{\textgreater}{\textgreater} }getleapsecs(x)
\pysrcoutput{32}\end{alltt}
    (section) Author:

      Josef Koller, Los Alamos National Lab (jkoller@lanl.gov)

    (section) Version:

      V1: 20-Jan-2010: includes array support

\setlength{\parskip}{1ex}
    \end{boxedminipage}

    \label{spacepy:ticktock:doy2date}
    \index{spacepy \textit{(package)}!spacepy.ticktock \textit{(module)}!spacepy.ticktock.doy2date \textit{(function)}}

    \vspace{0.5ex}

\hspace{.8\funcindent}\begin{boxedminipage}{\funcwidth}

    \raggedright \textbf{doy2date}(\textit{year}, \textit{doy})

    \vspace{-1.5ex}

    \rule{\textwidth}{0.5\fboxrule}
\setlength{\parskip}{2ex}
    convert day-of-year doy into a month and day 
    http://pleac.sourceforge.net/pleac\_python/datesandtimes.html

    (section) Input:

      \begin{itemize}
      \setlength{\parskip}{0.6ex}
        \item year (int or array of int) : year

        \item doy (int or array of int) : day of year

      \end{itemize}

    (section) Returns:

      \begin{itemize}
      \setlength{\parskip}{0.6ex}
        \item year (int or array of int): year as integer number

        \item month (int or array of int) : month as integer number

        \item day (int or array of int) : as integer number

      \end{itemize}

    (section) Example:

\begin{alltt}
\pysrcprompt{{\textgreater}{\textgreater}{\textgreater} }year, month, day = doy2date(2002, 186)\end{alltt}
    (section) See also:

      date2doy

    (section) Author:

      Josef Koller, Los Alamos National Lab, jkoller@lanl.gov

    (section) Version:

      V1: 24-Jan-2010: can handle arrays as input

\setlength{\parskip}{1ex}
    \end{boxedminipage}


%%%%%%%%%%%%%%%%%%%%%%%%%%%%%%%%%%%%%%%%%%%%%%%%%%%%%%%%%%%%%%%%%%%%%%%%%%%
%%                           Class Description                           %%
%%%%%%%%%%%%%%%%%%%%%%%%%%%%%%%%%%%%%%%%%%%%%%%%%%%%%%%%%%%%%%%%%%%%%%%%%%%

    \index{spacepy \textit{(package)}!spacepy.ticktock \textit{(module)}!spacepy.ticktock.ticktock \textit{(class)}|(}
\subsection{Class ticktock}

    \label{spacepy:ticktock:ticktock}
\begin{alltt}

    pyti class holding various time coordinate systems (TAI, UTC, ISO, JD, MJD, UNX, RTD)

    Possible data types:
    ISO: ISO standard format like '2002-02-25T12:20:30'
    UTC: datetime object with UTC time
    TAI: elapsed seconds since 1958/1/1 (includes leap seconds)
    UNX: elapsed seconds since 1970/1/1 (all days have 86400 secs of occasionally unequal lenghts)
    JD: Julian days elapsed
    MJD: Modified Julian days
    RTD: Rata Die days elapsed since 1/1/1

    Input:
    ======
        - data (int, datetime, float, string) : time stamp
        - dtype (string) : data type for data ISO, UTC, TAI, UNX, JD, MJD, RTD

    Returns:
    ========
        - object with self.[TAI, ISO, UTC, UNX, MJD, JD, RTD, DOY] 

    Example:
    ========

    {\textgreater}{\textgreater}{\textgreater} x=ticktock(2452331.0142361112, 'JD')
    {\textgreater}{\textgreater}{\textgreater} x.ISO
    '2002-02-25T12:20:30'
    {\textgreater}{\textgreater}{\textgreater} x.DOY \# Day of year
    56

x.    Author:
    =======
    Josef Koller, Los Alamos National Lab (jkoller@lanl.gov)

    Version:
    ========
    V1: 20-Jan-2010
    V2: 25-Jan-2010: includes array support (JK)
    
\end{alltt}


%%%%%%%%%%%%%%%%%%%%%%%%%%%%%%%%%%%%%%%%%%%%%%%%%%%%%%%%%%%%%%%%%%%%%%%%%%%
%%                                Methods                                %%
%%%%%%%%%%%%%%%%%%%%%%%%%%%%%%%%%%%%%%%%%%%%%%%%%%%%%%%%%%%%%%%%%%%%%%%%%%%

  \subsubsection{Methods}

    \label{spacepy:ticktock:ticktock:__init__}
    \index{spacepy \textit{(package)}!spacepy.ticktock \textit{(module)}!spacepy.ticktock.ticktock \textit{(class)}!spacepy.ticktock.ticktock.\_\_init\_\_ \textit{(method)}}

    \vspace{0.5ex}

\hspace{.8\funcindent}\begin{boxedminipage}{\funcwidth}

    \raggedright \textbf{\_\_init\_\_}(\textit{self}, \textit{data}, \textit{dtype})

\setlength{\parskip}{2ex}
\setlength{\parskip}{1ex}
    \end{boxedminipage}

    \index{spacepy \textit{(package)}!spacepy.ticktock \textit{(module)}!spacepy.ticktock.ticktock \textit{(class)}|)}
    \index{spacepy \textit{(package)}!spacepy.ticktock \textit{(module)}|)}

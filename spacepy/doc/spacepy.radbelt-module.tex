%
% API Documentation for SpacePy: Space Science Tools for Python
% Module spacepy.radbelt
%
% Generated by epydoc 3.0.1
% [Mon May 17 14:36:32 2010]
%

%%%%%%%%%%%%%%%%%%%%%%%%%%%%%%%%%%%%%%%%%%%%%%%%%%%%%%%%%%%%%%%%%%%%%%%%%%%
%%                          Module Description                           %%
%%%%%%%%%%%%%%%%%%%%%%%%%%%%%%%%%%%%%%%%%%%%%%%%%%%%%%%%%%%%%%%%%%%%%%%%%%%

    \index{spacepy \textit{(package)}!spacepy.radbelt \textit{(module)}|(}
\section{Module spacepy.radbelt}

    \label{spacepy:radbelt}
Functions supporting radiation belt diffusion codes

\textbf{Version:} \$Revision: 1.1.1.1 $, \$Date: 2010/05/20 17:19:45 $



\textbf{Author:} J. Koller, Los Alamos National Lab (jkoller@lanl.gov)




%%%%%%%%%%%%%%%%%%%%%%%%%%%%%%%%%%%%%%%%%%%%%%%%%%%%%%%%%%%%%%%%%%%%%%%%%%%
%%                               Functions                               %%
%%%%%%%%%%%%%%%%%%%%%%%%%%%%%%%%%%%%%%%%%%%%%%%%%%%%%%%%%%%%%%%%%%%%%%%%%%%

  \subsection{Functions}

    \label{spacepy:radbelt:diff_LL}
    \index{spacepy \textit{(package)}!spacepy.radbelt \textit{(module)}!spacepy.radbelt.diff\_LL \textit{(function)}}

    \vspace{0.5ex}

\hspace{.8\funcindent}\begin{boxedminipage}{\funcwidth}

    \raggedright \textbf{diff\_LL}(\textit{grid}, \textit{f}, \textit{Tdelta}, \textit{params}={\tt None})

    \vspace{-1.5ex}

    \rule{\textwidth}{0.5\fboxrule}
\setlength{\parskip}{2ex}
    calculate the diffusion in L at constant mu,K coordinates time units

\setlength{\parskip}{1ex}
    \end{boxedminipage}

    \label{spacepy:radbelt:get_modelop_L}
    \index{spacepy \textit{(package)}!spacepy.radbelt \textit{(module)}!spacepy.radbelt.get\_modelop\_L \textit{(function)}}

    \vspace{0.5ex}

\hspace{.8\funcindent}\begin{boxedminipage}{\funcwidth}

    \raggedright \textbf{get\_modelop\_L}(\textit{Lgrid}, \textit{Tdelta}, \textit{DLL})

    \vspace{-1.5ex}

    \rule{\textwidth}{0.5\fboxrule}
\setlength{\parskip}{2ex}
    calculate the model oparator for a single small timestep

\setlength{\parskip}{1ex}
    \end{boxedminipage}

    \label{spacepy:radbelt:get_DLL}
    \index{spacepy \textit{(package)}!spacepy.radbelt \textit{(module)}!spacepy.radbelt.get\_DLL \textit{(function)}}

    \vspace{0.5ex}

\hspace{.8\funcindent}\begin{boxedminipage}{\funcwidth}

    \raggedright \textbf{get\_DLL}(\textit{Lgrid}, \textit{params}, \textit{DLL\_model}={\tt \texttt{'}\texttt{BA2000}\texttt{'}})

    \vspace{-1.5ex}

    \rule{\textwidth}{0.5\fboxrule}
\setlength{\parskip}{2ex}
    calculate the diffusion coefficient D\_LL

\setlength{\parskip}{1ex}
    \end{boxedminipage}

    \label{spacepy:radbelt:get_local_accel}
    \index{spacepy \textit{(package)}!spacepy.radbelt \textit{(module)}!spacepy.radbelt.get\_local\_accel \textit{(function)}}

    \vspace{0.5ex}

\hspace{.8\funcindent}\begin{boxedminipage}{\funcwidth}

    \raggedright \textbf{get\_local\_accel}(\textit{Lgrid}, \textit{params}, \textit{SRC\_model}={\tt \texttt{'}\texttt{JK1}\texttt{'}})

    \vspace{-1.5ex}

    \rule{\textwidth}{0.5\fboxrule}
\setlength{\parskip}{2ex}
    calculate the diffusion coefficient D\_LL

\setlength{\parskip}{1ex}
    \end{boxedminipage}


%%%%%%%%%%%%%%%%%%%%%%%%%%%%%%%%%%%%%%%%%%%%%%%%%%%%%%%%%%%%%%%%%%%%%%%%%%%
%%                               Variables                               %%
%%%%%%%%%%%%%%%%%%%%%%%%%%%%%%%%%%%%%%%%%%%%%%%%%%%%%%%%%%%%%%%%%%%%%%%%%%%

  \subsection{Variables}

    \vspace{-1cm}
\hspace{\varindent}\begin{longtable}{|p{\varnamewidth}|p{\vardescrwidth}|l}
\cline{1-2}
\cline{1-2} \centering \textbf{Name} & \centering \textbf{Description}& \\
\cline{1-2}
\endhead\cline{1-2}\multicolumn{3}{r}{\small\textit{continued on next page}}\\\endfoot\cline{1-2}
\endlastfoot\raggedright \_\-\_\-l\-o\-g\-\_\-\_\- & \raggedright \textbf{Value:} 
{\tt \texttt{'}\texttt{{\textbackslash}n\$Log: spacepy.radbelt-module.tex,v $
{\tt \texttt{'}\texttt{{\textbackslash}n\Revision 1.1.1.1  2010/05/20 17:19:45  smorley
{\tt \texttt{'}\texttt{{\textbackslash}n\Pre-release repo import for SpacePy
{\tt \texttt{'}\texttt{{\textbackslash}n\
{\tt \texttt{'}\texttt{{\textbackslash}n\Revision 1.1  2010/05/19 22:30:51  smorley
{\tt \texttt{'}\texttt{{\textbackslash}n\Regenerated documentation
{\tt \texttt{'}\texttt{{\textbackslash}n\{\textbackslash}nRevision 1.7  2010/05/17 17:38:5}\texttt{...}}&\\
\cline{1-2}
\raggedright \_\-\_\-p\-a\-c\-k\-a\-g\-e\-\_\-\_\- & \raggedright \textbf{Value:} 
{\tt \texttt{'}\texttt{spacepy}\texttt{'}}&\\
\cline{1-2}
\end{longtable}


%%%%%%%%%%%%%%%%%%%%%%%%%%%%%%%%%%%%%%%%%%%%%%%%%%%%%%%%%%%%%%%%%%%%%%%%%%%
%%                           Class Description                           %%
%%%%%%%%%%%%%%%%%%%%%%%%%%%%%%%%%%%%%%%%%%%%%%%%%%%%%%%%%%%%%%%%%%%%%%%%%%%

    \index{spacepy \textit{(package)}!spacepy.radbelt \textit{(module)}!spacepy.radbelt.RBmodel \textit{(class)}|(}
\subsection{Class RBmodel}

    \label{spacepy:radbelt:RBmodel}
\begin{tabular}{cccccc}
% Line for object, linespec=[False]
\multicolumn{2}{r}{\settowidth{\BCL}{object}\multirow{2}{\BCL}{object}}
&&
  \\\cline{3-3}
  &&\multicolumn{1}{c|}{}
&&
  \\
&&\multicolumn{2}{l}{\textbf{spacepy.radbelt.RBmodel}}
\end{tabular}

1-D Radial diffusion class

This module contains a class for performing and visualizing 1-D radial 
diffusion simulations of the radiation belts.

Here is an example using the default settings of the model. Each instance 
must be initialized with (assuming import radbelt as rb):

\begin{alltt}
\pysrcprompt{{\textgreater}{\textgreater}{\textgreater} }rmod = rb.RBmodel()\end{alltt}
Next, set the start time, end time, and the size of the timestep:

\begin{alltt}
\pysrcprompt{{\textgreater}{\textgreater}{\textgreater} }start = datetime.datetime(2003,10,14)
\pysrcprompt{{\textgreater}{\textgreater}{\textgreater} }end = datetime.datetime(2003,12,26)
\pysrcprompt{{\textgreater}{\textgreater}{\textgreater} }delta = datetime.timedelta(hours=1)
\pysrcprompt{{\textgreater}{\textgreater}{\textgreater} }rmod.setup\_ticks(start, end, delta, dtype=\pysrcstring{'UTC'})\end{alltt}
Now, run the model over the enitre time range using the evolve method:

\begin{alltt}
\pysrcprompt{{\textgreater}{\textgreater}{\textgreater} }rmod.evolve()\end{alltt}
Finally, visualize the results:

\begin{alltt}
\pysrcprompt{{\textgreater}{\textgreater}{\textgreater} }rmod.plot\_summary()\end{alltt}
(section) Author:

  Josef Koller, Los Alamos National Lab (jkoller@lanl.gov)

(section) Version:

  V1: 17-Mar-2010 (JK) v1.01 12-May-2010 (dtw)


%%%%%%%%%%%%%%%%%%%%%%%%%%%%%%%%%%%%%%%%%%%%%%%%%%%%%%%%%%%%%%%%%%%%%%%%%%%
%%                                Methods                                %%
%%%%%%%%%%%%%%%%%%%%%%%%%%%%%%%%%%%%%%%%%%%%%%%%%%%%%%%%%%%%%%%%%%%%%%%%%%%

  \subsubsection{Methods}

    \vspace{0.5ex}

\hspace{.8\funcindent}\begin{boxedminipage}{\funcwidth}

    \raggedright \textbf{\_\_init\_\_}(\textit{self}, \textit{grid}={\tt \texttt{'}\texttt{L}\texttt{'}})

    \vspace{-1.5ex}

    \rule{\textwidth}{0.5\fboxrule}
\setlength{\parskip}{2ex}
    format for grid e.g., L-PA-E

\setlength{\parskip}{1ex}
      Overrides: object.\_\_init\_\_

    \end{boxedminipage}

    \vspace{0.5ex}

\hspace{.8\funcindent}\begin{boxedminipage}{\funcwidth}

    \raggedright \textbf{\_\_str\_\_}(\textit{self})

\setlength{\parskip}{2ex}
    str(x)

\setlength{\parskip}{1ex}
      Overrides: object.\_\_str\_\_

    \end{boxedminipage}

    \vspace{0.5ex}

\hspace{.8\funcindent}\begin{boxedminipage}{\funcwidth}

    \raggedright \textbf{\_\_repr\_\_}(\textit{self})

\setlength{\parskip}{2ex}
    str(x)

\setlength{\parskip}{1ex}
      Overrides: object.\_\_repr\_\_

    \end{boxedminipage}

    \label{spacepy:radbelt:RBmodel:__getitem__}
    \index{spacepy \textit{(package)}!spacepy.radbelt \textit{(module)}!spacepy.radbelt.RBmodel \textit{(class)}!spacepy.radbelt.RBmodel.\_\_getitem\_\_ \textit{(method)}}

    \vspace{0.5ex}

\hspace{.8\funcindent}\begin{boxedminipage}{\funcwidth}

    \raggedright \textbf{\_\_getitem\_\_}(\textit{self}, \textit{idx})

\setlength{\parskip}{2ex}
\setlength{\parskip}{1ex}
    \end{boxedminipage}

    \label{spacepy:radbelt:RBmodel:setup_ticks}
    \index{spacepy \textit{(package)}!spacepy.radbelt \textit{(module)}!spacepy.radbelt.RBmodel \textit{(class)}!spacepy.radbelt.RBmodel.setup\_ticks \textit{(method)}}

    \vspace{0.5ex}

\hspace{.8\funcindent}\begin{boxedminipage}{\funcwidth}

    \raggedright \textbf{setup\_ticks}(\textit{self}, \textit{start}, \textit{end}, \textit{delta}, \textit{dtype}={\tt \texttt{'}\texttt{ISO}\texttt{'}})

    \vspace{-1.5ex}

    \rule{\textwidth}{0.5\fboxrule}
\setlength{\parskip}{2ex}
    add time TickTock information to class instance

\setlength{\parskip}{1ex}
    \end{boxedminipage}

    \label{spacepy:radbelt:RBmodel:add_omni}
    \index{spacepy \textit{(package)}!spacepy.radbelt \textit{(module)}!spacepy.radbelt.RBmodel \textit{(class)}!spacepy.radbelt.RBmodel.add\_omni \textit{(method)}}

    \vspace{0.5ex}

\hspace{.8\funcindent}\begin{boxedminipage}{\funcwidth}

    \raggedright \textbf{add\_omni}(\textit{self}, \textit{keylist}={\tt None})

    \vspace{-1.5ex}

    \rule{\textwidth}{0.5\fboxrule}
\setlength{\parskip}{2ex}
    add omni data to instance according to the tickrange in ticktock

\setlength{\parskip}{1ex}
    \end{boxedminipage}

    \label{spacepy:radbelt:RBmodel:add_Lmax}
    \index{spacepy \textit{(package)}!spacepy.radbelt \textit{(module)}!spacepy.radbelt.RBmodel \textit{(class)}!spacepy.radbelt.RBmodel.add\_Lmax \textit{(method)}}

    \vspace{0.5ex}

\hspace{.8\funcindent}\begin{boxedminipage}{\funcwidth}

    \raggedright \textbf{add\_Lmax}(\textit{self}, \textit{Lmax\_model})

    \vspace{-1.5ex}

    \rule{\textwidth}{0.5\fboxrule}
\setlength{\parskip}{2ex}
    add last closed drift shell Lmax

\setlength{\parskip}{1ex}
    \end{boxedminipage}

    \label{spacepy:radbelt:RBmodel:add_Lpp}
    \index{spacepy \textit{(package)}!spacepy.radbelt \textit{(module)}!spacepy.radbelt.RBmodel \textit{(class)}!spacepy.radbelt.RBmodel.add\_Lpp \textit{(method)}}

    \vspace{0.5ex}

\hspace{.8\funcindent}\begin{boxedminipage}{\funcwidth}

    \raggedright \textbf{add\_Lpp}(\textit{self}, \textit{Lpp\_model})

    \vspace{-1.5ex}

    \rule{\textwidth}{0.5\fboxrule}
\setlength{\parskip}{2ex}
    add last closed drift shell Lmax

\setlength{\parskip}{1ex}
    \end{boxedminipage}

    \label{spacepy:radbelt:RBmodel:add_obs}
    \index{spacepy \textit{(package)}!spacepy.radbelt \textit{(module)}!spacepy.radbelt.RBmodel \textit{(class)}!spacepy.radbelt.RBmodel.add\_obs \textit{(method)}}

    \vspace{0.5ex}

\hspace{.8\funcindent}\begin{boxedminipage}{\funcwidth}

    \raggedright \textbf{add\_obs}(\textit{self}, \textit{satlist}={\tt None})

    \vspace{-1.5ex}

    \rule{\textwidth}{0.5\fboxrule}
\setlength{\parskip}{2ex}
    add observations from PSDdb using the ticktock list

\setlength{\parskip}{1ex}
    \end{boxedminipage}

    \label{spacepy:radbelt:RBmodel:evolve}
    \index{spacepy \textit{(package)}!spacepy.radbelt \textit{(module)}!spacepy.radbelt.RBmodel \textit{(class)}!spacepy.radbelt.RBmodel.evolve \textit{(method)}}

    \vspace{0.5ex}

\hspace{.8\funcindent}\begin{boxedminipage}{\funcwidth}

    \raggedright \textbf{evolve}(\textit{self})

    \vspace{-1.5ex}

    \rule{\textwidth}{0.5\fboxrule}
\setlength{\parskip}{2ex}
    calculate the diffusion in L at constant mu,K coordinates

\setlength{\parskip}{1ex}
    \end{boxedminipage}

    \label{spacepy:radbelt:RBmodel:assimilate}
    \index{spacepy \textit{(package)}!spacepy.radbelt \textit{(module)}!spacepy.radbelt.RBmodel \textit{(class)}!spacepy.radbelt.RBmodel.assimilate \textit{(method)}}

    \vspace{0.5ex}

\hspace{.8\funcindent}\begin{boxedminipage}{\funcwidth}

    \raggedright \textbf{assimilate}(\textit{self}, \textit{method}={\tt \texttt{'}\texttt{enKF}\texttt{'}})

    \vspace{-1.5ex}

    \rule{\textwidth}{0.5\fboxrule}
\setlength{\parskip}{2ex}
    call data assimilation function in assimilate.py

\setlength{\parskip}{1ex}
    \end{boxedminipage}

    \label{spacepy:radbelt:RBmodel:plot}
    \index{spacepy \textit{(package)}!spacepy.radbelt \textit{(module)}!spacepy.radbelt.RBmodel \textit{(class)}!spacepy.radbelt.RBmodel.plot \textit{(method)}}

    \vspace{0.5ex}

\hspace{.8\funcindent}\begin{boxedminipage}{\funcwidth}

    \raggedright \textbf{plot}(\textit{self}, \textit{Lmax}={\tt True}, \textit{Lpp}={\tt False}, \textit{Kp}={\tt True}, \textit{Dst}={\tt True}, \textit{clims}={\tt \texttt{[}0\texttt{, }10\texttt{]}}, \textit{title}={\tt \texttt{'}\texttt{Summary Plot}\texttt{'}})

    \vspace{-1.5ex}

    \rule{\textwidth}{0.5\fboxrule}
\setlength{\parskip}{2ex}
    Create a summary plot of the RadBelt object distribution function. For 
    reference, the last closed drift shell, Dst, and Kp are all included.  
    These can be disabled individually using the corresponding boolean 
    kwargs.

    The clims kwarg can be used to manually set the color bar range. To 
    use, set it equal to a two-element list containing minimum and maximum 
    Log\_10 value to plot.  Default action is to use [0,10] as the log\_10 
    of the color range.  This is good enough for most applications.

    The title of the top most plot defaults to 'Summary Plot' but can be 
    customized using the title kwarg.

    The figure object and all three axis objects (PSD axis, Dst axis, and 
    Kp axis) are all returned to allow the user to further customize the 
    plots as necessary.  If any of the plots are excluded, None is returned
    in their stead.

    Example:

\begin{alltt}
\pysrcprompt{{\textgreater}{\textgreater}{\textgreater} }rb.plot(Lmax=False, Kp=False, clims=[2,10], title=\pysrcstring{'Good work!'})\end{alltt}
    This command would create the summary plot with a color bar range of 
    100 to 10{\textasciicircum}10.  The Lmax line and Kp values would be 
    excluded. The title of the topmost plot (phase space density) would be 
    set to 'Good work!'.

\setlength{\parskip}{1ex}
    \end{boxedminipage}


\large{\textbf{\textit{Inherited from object}}}

\begin{quote}
\_\_delattr\_\_(), \_\_format\_\_(), \_\_getattribute\_\_(), \_\_hash\_\_(), \_\_new\_\_(), \_\_reduce\_\_(), \_\_reduce\_ex\_\_(), \_\_setattr\_\_(), \_\_sizeof\_\_(), \_\_subclasshook\_\_()
\end{quote}

%%%%%%%%%%%%%%%%%%%%%%%%%%%%%%%%%%%%%%%%%%%%%%%%%%%%%%%%%%%%%%%%%%%%%%%%%%%
%%                              Properties                               %%
%%%%%%%%%%%%%%%%%%%%%%%%%%%%%%%%%%%%%%%%%%%%%%%%%%%%%%%%%%%%%%%%%%%%%%%%%%%

  \subsubsection{Properties}

    \vspace{-1cm}
\hspace{\varindent}\begin{longtable}{|p{\varnamewidth}|p{\vardescrwidth}|l}
\cline{1-2}
\cline{1-2} \centering \textbf{Name} & \centering \textbf{Description}& \\
\cline{1-2}
\endhead\cline{1-2}\multicolumn{3}{r}{\small\textit{continued on next page}}\\\endfoot\cline{1-2}
\endlastfoot\multicolumn{2}{|l|}{\textit{Inherited from object}}\\
\multicolumn{2}{|p{\varwidth}|}{\raggedright \_\_class\_\_}\\
\cline{1-2}
\end{longtable}

    \index{spacepy \textit{(package)}!spacepy.radbelt \textit{(module)}!spacepy.radbelt.RBmodel \textit{(class)}|)}
    \index{spacepy \textit{(package)}!spacepy.radbelt \textit{(module)}|)}

%
% API Documentation for SpacePy: Space Science Tools for Python
% Module spacepy.toolbox
%
% Generated by epydoc 3.0.1
% [Mon May 17 14:36:32 2010]
%

%%%%%%%%%%%%%%%%%%%%%%%%%%%%%%%%%%%%%%%%%%%%%%%%%%%%%%%%%%%%%%%%%%%%%%%%%%%
%%                          Module Description                           %%
%%%%%%%%%%%%%%%%%%%%%%%%%%%%%%%%%%%%%%%%%%%%%%%%%%%%%%%%%%%%%%%%%%%%%%%%%%%

    \index{spacepy \textit{(package)}!spacepy.toolbox \textit{(module)}|(}
\section{Module spacepy.toolbox}

    \label{spacepy:toolbox}
Toolbox of various functions and generic utilities.

\textbf{Version:} \$Revision: 1.1 $, \$Date: 2010/05/20 17:19:45 $



\textbf{Author:} S. Morley and J. Koller




%%%%%%%%%%%%%%%%%%%%%%%%%%%%%%%%%%%%%%%%%%%%%%%%%%%%%%%%%%%%%%%%%%%%%%%%%%%
%%                               Functions                               %%
%%%%%%%%%%%%%%%%%%%%%%%%%%%%%%%%%%%%%%%%%%%%%%%%%%%%%%%%%%%%%%%%%%%%%%%%%%%

  \subsection{Functions}

    \label{spacepy:toolbox:doy2md}
    \index{spacepy \textit{(package)}!spacepy.toolbox \textit{(module)}!spacepy.toolbox.doy2md \textit{(function)}}

    \vspace{0.5ex}

\hspace{.8\funcindent}\begin{boxedminipage}{\funcwidth}

    \raggedright \textbf{doy2md}(\textit{year}, \textit{doy})

    \vspace{-1.5ex}

    \rule{\textwidth}{0.5\fboxrule}
\setlength{\parskip}{2ex}
    Convert day-of-year to month and day

    (section) Inputs:

      Year, Day of year (Jan 1 = 001)

    (section) Returns:

      Month, Day (e.g. Oct 11 = 10, 11)

      Note: Implements full and correct leap year rules.

    (section) Author:

      Steve Morley, Los Alamos National Lab, 
      smorley@lanl.gov/morley\_steve@hotmail.com

      Modification history: ==================== Created by Steve Morley in
      July '05, rewritten for Python in October 2009

\setlength{\parskip}{1ex}
    \end{boxedminipage}

    \label{spacepy:toolbox:sec2hms}
    \index{spacepy \textit{(package)}!spacepy.toolbox \textit{(module)}!spacepy.toolbox.sec2hms \textit{(function)}}

    \vspace{0.5ex}

\hspace{.8\funcindent}\begin{boxedminipage}{\funcwidth}

    \raggedright \textbf{sec2hms}(\textit{sec}, \textit{rounding}={\tt True}, \textit{days}={\tt False}, \textit{dtobj}={\tt False})

    \vspace{-1.5ex}

    \rule{\textwidth}{0.5\fboxrule}
\setlength{\parskip}{2ex}
\begin{alltt}
Convert seconds of day to hours, minutes, seconds

Inputs:
=======
Seconds of day
Keyword arguments:
    rounding (True{\textbar}False) - set for integer seconds
    days (True{\textbar}False) - set to wrap around day (i.e. modulo 86400)
    dtobj (True{\textbar}False) - set to return a timedelta object

Returns:
========
[hours, minutes, seconds] or datetime.timedelta

Author:
=======
Steve Morley, Los Alamos National Lab, smorley@lanl.gov/morley\_steve@hotmail.com

Modification history:
====================
v1. Created by Steve Morley in March 2010
v1.1 Datetime timedelta output added; 17-May-2010 (SM)
\end{alltt}

\setlength{\parskip}{1ex}
    \end{boxedminipage}

    \label{spacepy:toolbox:tOverlap}
    \index{spacepy \textit{(package)}!spacepy.toolbox \textit{(module)}!spacepy.toolbox.tOverlap \textit{(function)}}

    \vspace{0.5ex}

\hspace{.8\funcindent}\begin{boxedminipage}{\funcwidth}

    \raggedright \textbf{tOverlap}(\textit{ts1}, \textit{ts2})

    \vspace{-1.5ex}

    \rule{\textwidth}{0.5\fboxrule}
\setlength{\parskip}{2ex}
    Finds the overlapping elements in two lists of datetime objects

    (section) Returns:

      \begin{itemize}
      \setlength{\parskip}{0.6ex}
        \item indices of 1 within interval of 2, \& vice versa

      \end{itemize}

    (section) Example:

      \begin{itemize}
      \setlength{\parskip}{0.6ex}
        \item Given two series of datetime objects, event\_dates and 
          omni['Time']:

      \end{itemize}

\begin{alltt}
\pysrcprompt{{\textgreater}{\textgreater}{\textgreater} }\pysrckeyword{import} spacepy.toolbox \pysrckeyword{as} tb
\pysrcprompt{{\textgreater}{\textgreater}{\textgreater} }[einds,oinds] = tb.tOverlap(event\_dates, omni[\pysrcstring{'Time'}])
\pysrcprompt{{\textgreater}{\textgreater}{\textgreater} }omni\_time = omni[\pysrcstring{'Time'}][oinds[0]:oinds[-1]+1]
\pysrcprompt{{\textgreater}{\textgreater}{\textgreater} }\pysrckeyword{print} omni\_time
\pysrcoutput{[datetime.datetime(2007, 5, 5, 17, 57, 30), datetime.datetime(2007, 5, 5, 18, 2, 30),}
\pysrcoutput{... , datetime.datetime(2007, 5, 10, 4, 57, 30)]}\end{alltt}
    (section) Author:

      Steve Morley, Los Alamos National Lab, 
      smorley@lanl.gov/morley\_steve@hotmail.com

    (section) Modifications:

      Apr-2010: Add sanity check so when there are no overlapping elements,
      returns tuple of Nones Apr-2010: Additional sanity check for data 
      equidistant from st and end pts

    (section) Notes:

      Can probably be rewritten to explicitly use datetime objects and keep
      method...

\setlength{\parskip}{1ex}
    \end{boxedminipage}

    \label{spacepy:toolbox:tCommon}
    \index{spacepy \textit{(package)}!spacepy.toolbox \textit{(module)}!spacepy.toolbox.tCommon \textit{(function)}}

    \vspace{0.5ex}

\hspace{.8\funcindent}\begin{boxedminipage}{\funcwidth}

    \raggedright \textbf{tCommon}(\textit{ts1}, \textit{ts2}, \textit{mask\_only}={\tt True})

    \vspace{-1.5ex}

    \rule{\textwidth}{0.5\fboxrule}
\setlength{\parskip}{2ex}
    Finds the elements in a list of datetime objects present in another

    (section) Returns:

      \begin{itemize}
      \setlength{\parskip}{0.6ex}
        \item Two element tuple of truth tables (of 1 present in 2, \& vice 
          versa)

      \end{itemize}

    (section) Author:

      Steve Morley, Los Alamos National Lab, 
      smorley@lanl.gov/morley\_steve@hotmail.com

\setlength{\parskip}{1ex}
    \end{boxedminipage}

    \label{spacepy:toolbox:loadpickle}
    \index{spacepy \textit{(package)}!spacepy.toolbox \textit{(module)}!spacepy.toolbox.loadpickle \textit{(function)}}

    \vspace{0.5ex}

\hspace{.8\funcindent}\begin{boxedminipage}{\funcwidth}

    \raggedright \textbf{loadpickle}(\textit{fln})

    \vspace{-1.5ex}

    \rule{\textwidth}{0.5\fboxrule}
\setlength{\parskip}{2ex}
    load a pickle and return content as dictionary

    (section) Input:

      \begin{itemize}
      \setlength{\parskip}{0.6ex}
        \item fln (string) : filename

      \end{itemize}

    (section) Returns:

      \begin{itemize}
      \setlength{\parskip}{0.6ex}
        \item d (dictionary) : dictionary with content from file

      \end{itemize}

    (section) Example:

\begin{alltt}
\pysrcprompt{{\textgreater}{\textgreater}{\textgreater} }d = loadpickle(\pysrcstring{'test.pbin'})\end{alltt}
    (section) See also:

      savepickle

    (section) Author:

      Josef Koller, Los Alamos National Lab (jkoller@lanl.gov)

    (section) Version:

      V1: 20-Jan-2010

\setlength{\parskip}{1ex}
    \end{boxedminipage}

    \label{spacepy:toolbox:savepickle}
    \index{spacepy \textit{(package)}!spacepy.toolbox \textit{(module)}!spacepy.toolbox.savepickle \textit{(function)}}

    \vspace{0.5ex}

\hspace{.8\funcindent}\begin{boxedminipage}{\funcwidth}

    \raggedright \textbf{savepickle}(\textit{fln}, \textit{dict})

    \vspace{-1.5ex}

    \rule{\textwidth}{0.5\fboxrule}
\setlength{\parskip}{2ex}
    save dictionary variable dict to a pickle with filename fln Author: 
    Josef Koller, jkoller@lanl.gov

    (section) Inputs:

      \begin{itemize}
      \setlength{\parskip}{0.6ex}
        \item fln (string) : filename

        \item dict (dictionary) : container with stuff

      \end{itemize}

    (section) Example:

\begin{alltt}
\pysrcprompt{{\textgreater}{\textgreater}{\textgreater} }d = \{\pysrcstring{'grade'}:[1,2,3], \pysrcstring{'name'}:[\pysrcstring{'Mary'}, \pysrcstring{'John'}, \pysrcstring{'Chris'}]\}
\pysrcprompt{{\textgreater}{\textgreater}{\textgreater} }savepickle(\pysrcstring{'test.pbin'}, d)\end{alltt}
    (section) See also:

      loadpickle

    (section) Author:

      Josef Koller, Los Alamos National Lab (jkoller@lanl.gov)

    (section) Version:

      V1: 20-Jan-2010

\setlength{\parskip}{1ex}
    \end{boxedminipage}

    \label{spacepy:toolbox:assemble}
    \index{spacepy \textit{(package)}!spacepy.toolbox \textit{(module)}!spacepy.toolbox.assemble \textit{(function)}}

    \vspace{0.5ex}

\hspace{.8\funcindent}\begin{boxedminipage}{\funcwidth}

    \raggedright \textbf{assemble}(\textit{fln\_pattern}, \textit{outfln})

    \vspace{-1.5ex}

    \rule{\textwidth}{0.5\fboxrule}
\setlength{\parskip}{2ex}
    assembles all pickled files matching fln\_pattern into single file and 
    save as outfln. Pattern may contain simple shell-style wildcards *? a 
    la fnmatch file will be assembled along time axis TAI, etc in 
    dictionary

    (section) Inputs:

      \begin{itemize}
      \setlength{\parskip}{0.6ex}
        \item fln\_pattern (string) : pattern to match filenames

        \item outfln (string) : filename to save combined files to

      \end{itemize}

    (section) Outputs:

      \begin{itemize}
      \setlength{\parskip}{0.6ex}
        \item dcomb (dict) : dictionary with combined values

      \end{itemize}

    (section) Example:

\begin{alltt}
\pysrcprompt{{\textgreater}{\textgreater}{\textgreater} }assemble(\pysrcstring{'input\_files\_*.pbin'}, \pysrcstring{'combined\_input.pbin'})
\pysrcoutput{adding input\_files\_2001.pbin}
\pysrcoutput{adding input\_files\_2002.pbin}
\pysrcoutput{adding input\_files\_2004.pbin}
\pysrcoutput{writing: combined\_input.pbin}\end{alltt}
    (section) Author:

      Josef Koller, Los Alamos National Lab, jkoller@lanl.gov

    (section) Version:

      V1: 20-Jan-2010

\setlength{\parskip}{1ex}
    \end{boxedminipage}

    \label{spacepy:toolbox:feq}
    \index{spacepy \textit{(package)}!spacepy.toolbox \textit{(module)}!spacepy.toolbox.feq \textit{(function)}}

    \vspace{0.5ex}

\hspace{.8\funcindent}\begin{boxedminipage}{\funcwidth}

    \raggedright \textbf{feq}(\textit{x}, \textit{y})

    \vspace{-1.5ex}

    \rule{\textwidth}{0.5\fboxrule}
\setlength{\parskip}{2ex}
    compare two floating point values if they are equal after: 
    http://www.lahey.com/float.htm

    further info at:

\begin{alltt}
   http://docs.python.org/tut/node16.html
   http://www.velocityreviews.com/forums/t351983-precision-for-equality-of-two-floats.html
   http://www.boost.org/libs/test/doc/components/test\_tools/floating\_point\_comparison.html
   http://howto.wikia.com/wiki/Howto\_compare\_float\_numbers\_in\_the\_C\_programming\_language\end{alltt}

    (section) Input:

      \begin{itemize}
      \setlength{\parskip}{0.6ex}
        \item x (float) : a number

        \item y (array of floats) : float value or array of floats

      \end{itemize}

    (section) Returns:

      \begin{itemize}
      \setlength{\parskip}{0.6ex}
        \item boolean value: True or False

      \end{itemize}

    (section) Example:

\begin{alltt}
\pysrcprompt{{\textgreater}{\textgreater}{\textgreater} }index = where( feq(Lpos,Lgrid) ) \pysrccomment{\# use float point comparison}\end{alltt}
    (section) Author:

      Josef Koller, Los Alamos National Lab, jkoller@lanl.gov

    (section) Version:

      V1: 20-Jan-2010

\setlength{\parskip}{1ex}
    \end{boxedminipage}

    \label{spacepy:toolbox:dictree}
    \index{spacepy \textit{(package)}!spacepy.toolbox \textit{(module)}!spacepy.toolbox.dictree \textit{(function)}}

    \vspace{0.5ex}

\hspace{.8\funcindent}\begin{boxedminipage}{\funcwidth}

    \raggedright \textbf{dictree}(\textit{in\_dict}, \textit{verbose}={\tt False}, \textit{spaces}={\tt None}, \textit{levels}={\tt True})

    \vspace{-1.5ex}

    \rule{\textwidth}{0.5\fboxrule}
\setlength{\parskip}{2ex}
    pretty print a dictionary tree

    (section) Input:

      \begin{itemize}
      \setlength{\parskip}{0.6ex}
        \item in\_dict (dictionary) : a complex dictionary (with substructures)

        \item spaces (string) : string will added for every line

        \item levels (default False) : number of levels to recurse through 
          (True means all)

        \item boolean value: True or False

      \end{itemize}

    (section) Example:

\begin{alltt}
\pysrcprompt{{\textgreater}{\textgreater}{\textgreater} }d = \{\pysrcstring{'grade'}:\{\pysrcstring{'level1'}:[4,5,6], \pysrcstring{'level2'}:[2,3,4]\}, \pysrcstring{'name'}:[\pysrcstring{'Mary'}, \pysrcstring{'John'}, \pysrcstring{'Chris'}]\}
\pysrcprompt{{\textgreater}{\textgreater}{\textgreater} }dictree(d)
\pysrcoutput{+}
\pysrcoutput{{\textbar}\_\_\_\_grade}
\pysrcoutput{    {\textbar}\_\_\_\_level1}
\pysrcoutput{    {\textbar}\_\_\_\_level2}
\pysrcoutput{{\textbar}\_\_\_\_name}\end{alltt}
    (section) Author:

      Josef Koller, Los Alamos National Lab, jkoller@lanl.gov

    (section) Version:

      V1: 20-Jan-2010 V1.1: 24-Feb-2010 S. Morley, added verbose option 
      v1.2: 17-May-2010 S. Morley, added levels option

\setlength{\parskip}{1ex}
    \end{boxedminipage}

    \label{spacepy:toolbox:printfig}
    \index{spacepy \textit{(package)}!spacepy.toolbox \textit{(module)}!spacepy.toolbox.printfig \textit{(function)}}

    \vspace{0.5ex}

\hspace{.8\funcindent}\begin{boxedminipage}{\funcwidth}

    \raggedright \textbf{printfig}(\textit{fignum}, \textit{saveonly}={\tt False}, \textit{pngonly}={\tt False}, \textit{clean}={\tt False})

    \vspace{-1.5ex}

    \rule{\textwidth}{0.5\fboxrule}
\setlength{\parskip}{2ex}
    save current figure to file and call lpr (print).

    This routine will create a total of 3 files (png, ps and c.png) in the 
    current working directory with a sequence number attached. Also, a time
    stamp and the location of the file will be imprinted on the figure. The
    file ending with c.png is clean and no directory or time stamp are 
    attached (good for powerpoint presentations).

    (section) Input:

      \begin{itemize}
      \setlength{\parskip}{0.6ex}
        \item fignum (integer or array/list of integer) : matplotlib figure 
          number

        \item optional

          \begin{itemize}
          \setlength{\parskip}{0.6ex}
            \item saveonly (boolean) : True (don't print and save only to file)
              False (print and save)

            \item pngonly (boolean) : True (only save png files and print png 
              directly) False (print ps file, and generate png, ps; can be 
              slow)

            \item clean (boolean) : True (print and save only clean files 
              without directory info) False (print and save directory 
              location as well)

          \end{itemize}

      \end{itemize}

    (section) Example:

\begin{alltt}
\pysrcprompt{{\textgreater}{\textgreater}{\textgreater} }pylab.plot([1,2,3],[2,3,2])
\pysrcprompt{{\textgreater}{\textgreater}{\textgreater} }spacepy.printfig(1)\end{alltt}
    (section) Author:

      Josef Koller, Los Alamos National Lab, jkoller@lanl.gov

    (section) Version:

      V1: 20-Jan-2010 V2: 19-Feb-2010: added pngonly and clean options, 
      array/list support (JK)

\setlength{\parskip}{1ex}
    \end{boxedminipage}

    \label{spacepy:toolbox:update}
    \index{spacepy \textit{(package)}!spacepy.toolbox \textit{(module)}!spacepy.toolbox.update \textit{(function)}}

    \vspace{0.5ex}

\hspace{.8\funcindent}\begin{boxedminipage}{\funcwidth}

    \raggedright \textbf{update}(\textit{all}={\tt True}, \textit{omni}={\tt False}, \textit{leapsecs}={\tt False})

    \vspace{-1.5ex}

    \rule{\textwidth}{0.5\fboxrule}
\setlength{\parskip}{2ex}
    Download and update local database for omni, leapsecs etc

    (section) Input:

      \begin{itemize}
      \setlength{\parskip}{0.6ex}
        \item all (bol) : if True, update all of them

        \item omni (bol) : if True. update only onmi

        \item leapsecs (bol) : if True, update only leapseconds

      \end{itemize}

    (section) Example:

\begin{alltt}
\pysrcprompt{{\textgreater}{\textgreater}{\textgreater} }\pysrcbuiltin{update}(omni=True)\end{alltt}
    (section) Author:

      Josef Koller, Los Alamos National Lab, jkoller@lanl.gov

    (section) Version:

      V1: 20-Jan-2010

\setlength{\parskip}{1ex}
    \end{boxedminipage}

    \label{spacepy:toolbox:windowMean}
    \index{spacepy \textit{(package)}!spacepy.toolbox \textit{(module)}!spacepy.toolbox.windowMean \textit{(function)}}

    \vspace{0.5ex}

\hspace{.8\funcindent}\begin{boxedminipage}{\funcwidth}

    \raggedright \textbf{windowMean}(\textit{data}, \textit{time}={\tt \texttt{[}\texttt{]}}, \textit{winsize}={\tt 0}, \textit{overlap}={\tt 0}, \textit{st\_time}={\tt None})

    \vspace{-1.5ex}

    \rule{\textwidth}{0.5\fboxrule}
\setlength{\parskip}{2ex}
    Windowing mean function, window overlap is user defined

    Inputs: data - 1D series of points; time - series of timestamps, 
    optional (format as numeric or datetime); For non-overlapping windows 
    set overlap to zero. e.g.,

\begin{alltt}
\pysrcprompt{{\textgreater}{\textgreater}{\textgreater} }wsize, olap = datetime.timedelta(1), datetime.timedelta(0,3600)\end{alltt}
\begin{alltt}
\pysrcprompt{{\textgreater}{\textgreater}{\textgreater} }outdata, outtime = windowmean(data, time, winsize=wsize, overlap=olap)\end{alltt}
    where the time, winsize and overlap are either numberic or datetime 
    objects, in this example the window size is 1 day and the overlap is 1 
    hour.

    Caveats: This is a quick and dirty function - it is NOT optimised, at 
    all.

    (section) Author:

      Steve Morley, Los Alamos National Lab, 
      smorley@lanl.gov/morley\_steve@hotmail.com

\setlength{\parskip}{1ex}
    \end{boxedminipage}

    \label{spacepy:toolbox:medAbsDev}
    \index{spacepy \textit{(package)}!spacepy.toolbox \textit{(module)}!spacepy.toolbox.medAbsDev \textit{(function)}}

    \vspace{0.5ex}

\hspace{.8\funcindent}\begin{boxedminipage}{\funcwidth}

    \raggedright \textbf{medAbsDev}(\textit{series})

    \vspace{-1.5ex}

    \rule{\textwidth}{0.5\fboxrule}
\setlength{\parskip}{2ex}
    Calculate median absolute deviation of a given input series

    Median absolute deviation (MAD) is a robust and resistant measure of 
    the spread of a sample (same purpose as standard deviation). The MAD is
    preferred to the interquartile range as the interquartile range only 
    shows 50\% of the data whereas the MAD uses all data but remains robust
    and resistant. See e.g. Wilks, Statistical methods for the Atmospheric 
    Sciences, 1995, Ch. 3.

    This implementation is robust to presence of NaNs

    Example: Find the median absolute deviation of a data set. Here we use 
    the log- normal distribution fitted to the population of sawtooth 
    intervals, see Morley and Henderson, Comment, Geophysical Research 
    Letters, 2009.

\begin{alltt}
\pysrcprompt{{\textgreater}{\textgreater}{\textgreater} }data = numpy.random.lognormal(mean=5.1458, sigma=0.302313, size=30)
\pysrcprompt{{\textgreater}{\textgreater}{\textgreater} }\pysrckeyword{print} data
\pysrcoutput{array([ 181.28078923,  131.18152745, ... , 141.15455416, 160.88972791])}
\pysrcoutput{}\pysrcprompt{{\textgreater}{\textgreater}{\textgreater} }utils.medabsdev(data)
\pysrcoutput{28.346646721370192}\end{alltt}
    (section) Author:

      Steve Morley, Los Alamos National Lab, 
      smorley@lanl.gov/morley\_steve@hotmail.com

\setlength{\parskip}{1ex}
    \end{boxedminipage}

    \label{spacepy:toolbox:makePoly}
    \index{spacepy \textit{(package)}!spacepy.toolbox \textit{(module)}!spacepy.toolbox.makePoly \textit{(function)}}

    \vspace{0.5ex}

\hspace{.8\funcindent}\begin{boxedminipage}{\funcwidth}

    \raggedright \textbf{makePoly}(\textit{x}, \textit{y1}, \textit{y2}, \textit{face}={\tt \texttt{'}\texttt{blue}\texttt{'}}, \textit{alpha}={\tt 0.5})

    \vspace{-1.5ex}

    \rule{\textwidth}{0.5\fboxrule}
\setlength{\parskip}{2ex}
    Make filled polygon for plotting

    Equivalent functionality to built-in matplotlib function fill\_between

\begin{alltt}
\pysrcprompt{{\textgreater}{\textgreater}{\textgreater} }poly0c = makePoly(x, ci\_low, ci\_high, face=\pysrcstring{'red'}, alpha=0.8)
\pysrcprompt{{\textgreater}{\textgreater}{\textgreater} }ax0.add\_patch(poly0qc)\end{alltt}
    (section) Author:

      Steve Morley, Los Alamos National Lab, 
      smorley@lanl.gov/morley\_steve@hotmail.com

\setlength{\parskip}{1ex}
    \end{boxedminipage}

    \label{spacepy:toolbox:binHisto}
    \index{spacepy \textit{(package)}!spacepy.toolbox \textit{(module)}!spacepy.toolbox.binHisto \textit{(function)}}

    \vspace{0.5ex}

\hspace{.8\funcindent}\begin{boxedminipage}{\funcwidth}

    \raggedright \textbf{binHisto}(\textit{data})

    \vspace{-1.5ex}

    \rule{\textwidth}{0.5\fboxrule}
\setlength{\parskip}{2ex}
    Calculates bin width and number of bins for histogram using 
    Freedman-Diaconis rule

    (section) Inputs:

      data - list/array of data values

    (section) Outputs:

      binw - calculated width of bins using F-D rule nbins - number of bins
      (nearest integer) to use for histogram

    (section) Example:

\begin{alltt}
\pysrcprompt{{\textgreater}{\textgreater}{\textgreater} }\pysrckeyword{import} numpy, spacepy
\pysrcprompt{{\textgreater}{\textgreater}{\textgreater} }\pysrckeyword{import} matplotlib.pyplot \pysrckeyword{as} plt
\pysrcprompt{{\textgreater}{\textgreater}{\textgreater} }data = numpy.random.randn(100)
\pysrcprompt{{\textgreater}{\textgreater}{\textgreater} }binw, nbins = spacepy.toolbox.binHisto(data)
\pysrcprompt{{\textgreater}{\textgreater}{\textgreater} }plt.hist(data, bins=nbins, histtype=\pysrcstring{'step'}, normed=True)\end{alltt}
    (section) Author:

      Steve Morley, Los Alamos National Lab, 
      smorley@lanl.gov/morley\_steve@hotmail.com

\setlength{\parskip}{1ex}
    \end{boxedminipage}

    \label{spacepy:toolbox:smartTimeTicks}
    \index{spacepy \textit{(package)}!spacepy.toolbox \textit{(module)}!spacepy.toolbox.smartTimeTicks \textit{(function)}}

    \vspace{0.5ex}

\hspace{.8\funcindent}\begin{boxedminipage}{\funcwidth}

    \raggedright \textbf{smartTimeTicks}(\textit{time})

    \vspace{-1.5ex}

    \rule{\textwidth}{0.5\fboxrule}
\setlength{\parskip}{2ex}
    Returns major ticks, minor ticks and format for time-based plots

    smartTimeTicks takes a list of datetime objects and uses the range to 
    calculate the best tick spacing and format.

    (section) Inputs:

      time - list of datetime objects

    (section) Outputs:

      Mtick - major ticks mtick - minor ticks fmt - format

    (section) Example:

      ? Meh.

    (section) Author:

      Dan Welling, Los Alamos National Lab, 
      dwelling@lanl.gov/dantwelling@gmail.com

\setlength{\parskip}{1ex}
    \end{boxedminipage}

    \label{spacepy:toolbox:tCommon}
    \index{spacepy \textit{(package)}!spacepy.toolbox \textit{(module)}!spacepy.toolbox.tCommon \textit{(function)}}

    \vspace{0.5ex}

\hspace{.8\funcindent}\begin{boxedminipage}{\funcwidth}

    \raggedright \textbf{t\_common}(\textit{ts1}, \textit{ts2}, \textit{mask\_only}={\tt True})

    \vspace{-1.5ex}

    \rule{\textwidth}{0.5\fboxrule}
\setlength{\parskip}{2ex}
    Finds the elements in a list of datetime objects present in another

    (section) Returns:

      \begin{itemize}
      \setlength{\parskip}{0.6ex}
        \item Two element tuple of truth tables (of 1 present in 2, \& vice 
          versa)

      \end{itemize}

    (section) Author:

      Steve Morley, Los Alamos National Lab, 
      smorley@lanl.gov/morley\_steve@hotmail.com

\setlength{\parskip}{1ex}
    \end{boxedminipage}

    \label{spacepy:toolbox:tOverlap}
    \index{spacepy \textit{(package)}!spacepy.toolbox \textit{(module)}!spacepy.toolbox.tOverlap \textit{(function)}}

    \vspace{0.5ex}

\hspace{.8\funcindent}\begin{boxedminipage}{\funcwidth}

    \raggedright \textbf{t\_overlap}(\textit{ts1}, \textit{ts2})

    \vspace{-1.5ex}

    \rule{\textwidth}{0.5\fboxrule}
\setlength{\parskip}{2ex}
    Finds the overlapping elements in two lists of datetime objects

    (section) Returns:

      \begin{itemize}
      \setlength{\parskip}{0.6ex}
        \item indices of 1 within interval of 2, \& vice versa

      \end{itemize}

    (section) Example:

      \begin{itemize}
      \setlength{\parskip}{0.6ex}
        \item Given two series of datetime objects, event\_dates and 
          omni['Time']:

      \end{itemize}

\begin{alltt}
\pysrcprompt{{\textgreater}{\textgreater}{\textgreater} }\pysrckeyword{import} spacepy.toolbox \pysrckeyword{as} tb
\pysrcprompt{{\textgreater}{\textgreater}{\textgreater} }[einds,oinds] = tb.tOverlap(event\_dates, omni[\pysrcstring{'Time'}])
\pysrcprompt{{\textgreater}{\textgreater}{\textgreater} }omni\_time = omni[\pysrcstring{'Time'}][oinds[0]:oinds[-1]+1]
\pysrcprompt{{\textgreater}{\textgreater}{\textgreater} }\pysrckeyword{print} omni\_time
\pysrcoutput{[datetime.datetime(2007, 5, 5, 17, 57, 30), datetime.datetime(2007, 5, 5, 18, 2, 30),}
\pysrcoutput{... , datetime.datetime(2007, 5, 10, 4, 57, 30)]}\end{alltt}
    (section) Author:

      Steve Morley, Los Alamos National Lab, 
      smorley@lanl.gov/morley\_steve@hotmail.com

    (section) Modifications:

      Apr-2010: Add sanity check so when there are no overlapping elements,
      returns tuple of Nones Apr-2010: Additional sanity check for data 
      equidistant from st and end pts

    (section) Notes:

      Can probably be rewritten to explicitly use datetime objects and keep
      method...

\setlength{\parskip}{1ex}
    \end{boxedminipage}

    \label{spacepy:toolbox:smartTimeTicks}
    \index{spacepy \textit{(package)}!spacepy.toolbox \textit{(module)}!spacepy.toolbox.smartTimeTicks \textit{(function)}}

    \vspace{0.5ex}

\hspace{.8\funcindent}\begin{boxedminipage}{\funcwidth}

    \raggedright \textbf{smart\_timeticks}(\textit{time})

    \vspace{-1.5ex}

    \rule{\textwidth}{0.5\fboxrule}
\setlength{\parskip}{2ex}
    Returns major ticks, minor ticks and format for time-based plots

    smartTimeTicks takes a list of datetime objects and uses the range to 
    calculate the best tick spacing and format.

    (section) Inputs:

      time - list of datetime objects

    (section) Outputs:

      Mtick - major ticks mtick - minor ticks fmt - format

    (section) Example:

      ? Meh.

    (section) Author:

      Dan Welling, Los Alamos National Lab, 
      dwelling@lanl.gov/dantwelling@gmail.com

\setlength{\parskip}{1ex}
    \end{boxedminipage}

    \label{spacepy:toolbox:logspace}
    \index{spacepy \textit{(package)}!spacepy.toolbox \textit{(module)}!spacepy.toolbox.logspace \textit{(function)}}

    \vspace{0.5ex}

\hspace{.8\funcindent}\begin{boxedminipage}{\funcwidth}

    \raggedright \textbf{logspace}(\textit{min}, \textit{max}, \textit{num}, **\textit{kwargs})

    \vspace{-1.5ex}

    \rule{\textwidth}{0.5\fboxrule}
\setlength{\parskip}{2ex}
    Returns log spaced bins.  Same as numpy logspace except the min and max
    are the ,min and max not log10(min) and log10(max)

    logspace(min, max, num)

    (section) Inputs:

      min - minimum value max - maximum value num - number of log spaced 
      bins

    (section) Outputs:

      num log spaced bins from min to max in a numpy array

    (section) Example:

      logspace(1, 100, 5) Out[2]: array([   1.        ,    3.16227766,   
      10.        ,   31.6227766 ,  100.        ])

    (section) Author:

      Brian Larsen, Los Alamos National Lab, balarsen@lanl.gov

\setlength{\parskip}{1ex}
    \end{boxedminipage}

    \label{spacepy:toolbox:arraybin}
    \index{spacepy \textit{(package)}!spacepy.toolbox \textit{(module)}!spacepy.toolbox.arraybin \textit{(function)}}

    \vspace{0.5ex}

\hspace{.8\funcindent}\begin{boxedminipage}{\funcwidth}

    \raggedright \textbf{arraybin}(\textit{array}, \textit{bins})

    \vspace{-1.5ex}

    \rule{\textwidth}{0.5\fboxrule}
\setlength{\parskip}{2ex}
\begin{alltt}
Given an array and a set of bins return the indices that are less than the 
smallest bin between each set and larger than the largest bin
bins should be sorted

Inputs:
=======
array - the input array to slice
bins - the bins to slice along (may be array or list)

Outputs:
========
list of indices 
     first element is less than fisrt bin and last bin is larger than last bin

Example:
========
arraybin(arange(10), [4.2])
Out[4]: [(array([0, 1, 2, 3, 4]),), (array([5, 6, 7, 8, 9]),)]

Author:
=======
Brian Larsen, Los Alamos National Lab, balarsen@lanl.gov
\end{alltt}

\setlength{\parskip}{1ex}
    \end{boxedminipage}

    \label{spacepy:toolbox:mlt2rad}
    \index{spacepy \textit{(package)}!spacepy.toolbox \textit{(module)}!spacepy.toolbox.mlt2rad \textit{(function)}}

    \vspace{0.5ex}

\hspace{.8\funcindent}\begin{boxedminipage}{\funcwidth}

    \raggedright \textbf{mlt2rad}(\textit{mlt}, \textit{midnight}={\tt False})

    \vspace{-1.5ex}

    \rule{\textwidth}{0.5\fboxrule}
\setlength{\parskip}{2ex}
    Convert mlt values to radians for polar plotting transform mlt angles 
    to radians from -pi to pi referenced from noon by default

    (section) Inputs:

      mlt - array of mlt values midnight=False - reference to midnioght 
      instead of noon

    (section) Outputs:

      array of radians

    (section) Example:

      mlt2rad(array([3,6,9,14,22])) Out[9]: array([-2.35619449, 
      -1.57079633, -0.78539816,  0.52359878,  2.61799388])

    (section) Author:

      Brian Larsen, Los Alamos National Lab, balarsen@lanl.gov

\setlength{\parskip}{1ex}
    \end{boxedminipage}

    \label{spacepy:toolbox:rad2mlt}
    \index{spacepy \textit{(package)}!spacepy.toolbox \textit{(module)}!spacepy.toolbox.rad2mlt \textit{(function)}}

    \vspace{0.5ex}

\hspace{.8\funcindent}\begin{boxedminipage}{\funcwidth}

    \raggedright \textbf{rad2mlt}(\textit{rad}, \textit{midnight}={\tt False})

    \vspace{-1.5ex}

    \rule{\textwidth}{0.5\fboxrule}
\setlength{\parskip}{2ex}
    Convert radian values to mlt transform radians from -pi to pi to mlt 
    referenced from noon by default

    (section) Inputs:

      rad - array of rad values midnight=False - reference to midnioght 
      instead of noon

    (section) Outputs:

      array of mlt

    (section) Example:

      rad2mlt(array([0,pi, pi/2.])) Out[8]: array([ 12.,  24.,  18.])

    (section) Author:

      Brian Larsen, Los Alamos National Lab, balarsen@lanl.gov

\setlength{\parskip}{1ex}
    \end{boxedminipage}

    \label{spacepy:toolbox:leap_year}
    \index{spacepy \textit{(package)}!spacepy.toolbox \textit{(module)}!spacepy.toolbox.leap\_year \textit{(function)}}

    \vspace{0.5ex}

\hspace{.8\funcindent}\begin{boxedminipage}{\funcwidth}

    \raggedright \textbf{leap\_year}(\textit{year}, \textit{numdays}={\tt False})

    \vspace{-1.5ex}

    \rule{\textwidth}{0.5\fboxrule}
\setlength{\parskip}{2ex}
\begin{alltt}
return an array of boolean leap year, 
a lot faster than the mod method that is normally seen

Inputs:
=======
year - array of years
numdays=False - optionally return the number of days in the year

Outputs:
========
an array of boolean leap year, or array of number of days

Example:
========
leap\_year(arange(15)+1998)
Out[10]: 
array([False, False,  True, False, False, False,  True, False, False,
   False,  True, False, False, False,  True], dtype=bool)

Author:
=======
Brian Larsen, Los Alamos National Lab, balarsen@lanl.gov
\end{alltt}

\setlength{\parskip}{1ex}
    \end{boxedminipage}

    \label{spacepy:toolbox:pmm}
    \index{spacepy \textit{(package)}!spacepy.toolbox \textit{(module)}!spacepy.toolbox.pmm \textit{(function)}}

    \vspace{0.5ex}

\hspace{.8\funcindent}\begin{boxedminipage}{\funcwidth}

    \raggedright \textbf{pmm}(\textit{a}, *\textit{b})

    \vspace{-1.5ex}

    \rule{\textwidth}{0.5\fboxrule}
\setlength{\parskip}{2ex}
    print min and max of input arrays

    (section) Inputs:

      a - input array *b - some additional number of arrays

    (section) Outputs:

      list of min, max for each array

    (section) Example:

      pmm(arange(10), arange(10)+3) Out[12]: [(0, 9), (3, 12)]

    (section) Author:

      Brian Larsen, Los Alamos National Lab, balarsen@lanl.gov

\setlength{\parskip}{1ex}
    \end{boxedminipage}

    \label{spacepy:toolbox:timestamp}
    \index{spacepy \textit{(package)}!spacepy.toolbox \textit{(module)}!spacepy.toolbox.timestamp \textit{(function)}}

    \vspace{0.5ex}

\hspace{.8\funcindent}\begin{boxedminipage}{\funcwidth}

    \raggedright \textbf{timestamp}(\textit{position}={\tt 1.=...}, \textit{0.01}={\tt ...}, \textit{size}={\tt 'xx-small'}, **\textit{kwargs})

    \vspace{-1.5ex}

    \rule{\textwidth}{0.5\fboxrule}
\setlength{\parskip}{2ex}
    print a timestamp on the current plot, vertical lower right

    (section) Inputs:

      (all optional) position - position for the timestamp size - text size
      draw - call draw to make sure it appears kwargs - other keywords to 
      axis.annotate

    (section) Outputs:

      timestamp written to the current plot

    (section) Example:

      plot(arange(11)) Out[13]: [{\textless}matplotlib.lines.Line2D object 
      at 0x49072b0{\textgreater}] timestamp()

    (section) Author:

      Brian Larsen, Los Alamos National Lab, balarsen@lanl.gov

\setlength{\parskip}{1ex}
    \end{boxedminipage}

    \label{spacepy:toolbox:query_yes_no}
    \index{spacepy \textit{(package)}!spacepy.toolbox \textit{(module)}!spacepy.toolbox.query\_yes\_no \textit{(function)}}

    \vspace{0.5ex}

\hspace{.8\funcindent}\begin{boxedminipage}{\funcwidth}

    \raggedright \textbf{query\_yes\_no}(\textit{question}, \textit{default}={\tt \texttt{'}\texttt{yes}\texttt{'}})

    \vspace{-1.5ex}

    \rule{\textwidth}{0.5\fboxrule}
\setlength{\parskip}{2ex}
\begin{alltt}
Ask a yes/no question via raw\_input() and return their answer.

"question" is a string that is presented to the user.
"default" is the presumed answer if the user just hits {\textless}Enter{\textgreater}.
    It must be "yes" (the default), "no" or None (meaning
    an answer is required of the user).

The "answer" return value is one of "yes" or "no".

Inputs:
=======
question - string that is the question to ask
default - the default answer (yes)

Outputs:
========
answer ('yes' or 'no')

Example:
======== 
query\_yes\_no('Ready to go?')
Ready to go? [Y/n] y
Out[17]: 'yes'


Author:
=======
Brian Larsen, Los Alamos National Lab, balarsen@lanl.gov
\end{alltt}

\setlength{\parskip}{1ex}
    \end{boxedminipage}


%%%%%%%%%%%%%%%%%%%%%%%%%%%%%%%%%%%%%%%%%%%%%%%%%%%%%%%%%%%%%%%%%%%%%%%%%%%
%%                               Variables                               %%
%%%%%%%%%%%%%%%%%%%%%%%%%%%%%%%%%%%%%%%%%%%%%%%%%%%%%%%%%%%%%%%%%%%%%%%%%%%

  \subsection{Variables}

    \vspace{-1cm}
\hspace{\varindent}\begin{longtable}{|p{\varnamewidth}|p{\vardescrwidth}|l}
\cline{1-2}
\cline{1-2} \centering \textbf{Name} & \centering \textbf{Description}& \\
\cline{1-2}
\endhead\cline{1-2}\multicolumn{3}{r}{\small\textit{continued on next page}}\\\endfoot\cline{1-2}
\endlastfoot\raggedright \_\-\_\-l\-o\-g\-\_\-\_\- & \raggedright \textbf{Value:} 
{\tt \texttt{'}\texttt{{\textbackslash}n\$Log: spacepy.toolbox-module.tex,v $
{\tt \texttt{'}\texttt{{\textbackslash}n\Revision 1.1  2010/05/20 17:19:45  smorley
{\tt \texttt{'}\texttt{{\textbackslash}n\Initial revision
{\tt \texttt{'}\texttt{{\textbackslash}n\
{\tt \texttt{'}\texttt{{\textbackslash}n\Revision 1.2  2010/05/19 22:30:52  smorley
{\tt \texttt{'}\texttt{{\textbackslash}n\Regenerated documentation
{\tt \texttt{'}\texttt{{\textbackslash}n\{\textbackslash}nRevision 1.43  2010/05/17 20:20:}\texttt{...}}&\\
\cline{1-2}
\raggedright \_\-\_\-p\-a\-c\-k\-a\-g\-e\-\_\-\_\- & \raggedright \textbf{Value:} 
{\tt \texttt{'}\texttt{spacepy}\texttt{'}}&\\
\cline{1-2}
\end{longtable}

    \index{spacepy \textit{(package)}!spacepy.toolbox \textit{(module)}|)}

%
% API Documentation for SpacePy: Space Science Tools for Python
% Module spacepy.utils
%
% Generated by epydoc 3.0.1
% [Mon May 17 14:36:32 2010]
%

%%%%%%%%%%%%%%%%%%%%%%%%%%%%%%%%%%%%%%%%%%%%%%%%%%%%%%%%%%%%%%%%%%%%%%%%%%%
%%                          Module Description                           %%
%%%%%%%%%%%%%%%%%%%%%%%%%%%%%%%%%%%%%%%%%%%%%%%%%%%%%%%%%%%%%%%%%%%%%%%%%%%

    \index{spacepy \textit{(package)}!spacepy.utils \textit{(module)}|(}
\section{Module spacepy.utils}

    \label{spacepy:utils}
Set of generic utilities.

--++-- By Steve Morley --++--

smorley@lanl.gov/morley\_steve@hotmail.com, Los Alamos National Laboratory,
ISR-1, PO Box 1663, Los Alamos, NM 87545


%%%%%%%%%%%%%%%%%%%%%%%%%%%%%%%%%%%%%%%%%%%%%%%%%%%%%%%%%%%%%%%%%%%%%%%%%%%
%%                               Functions                               %%
%%%%%%%%%%%%%%%%%%%%%%%%%%%%%%%%%%%%%%%%%%%%%%%%%%%%%%%%%%%%%%%%%%%%%%%%%%%

  \subsection{Functions}

    \label{spacepy:utils:doy2md}
    \index{spacepy \textit{(package)}!spacepy.utils \textit{(module)}!spacepy.utils.doy2md \textit{(function)}}

    \vspace{0.5ex}

\hspace{.8\funcindent}\begin{boxedminipage}{\funcwidth}

    \raggedright \textbf{doy2md}(\textit{year}, \textit{doy})

    \vspace{-1.5ex}

    \rule{\textwidth}{0.5\fboxrule}
\setlength{\parskip}{2ex}
    Convert day-of-year to month and day

    (section) Inputs:

      Year, Day of year (Jan 1 = 001)

    (section) Returns:

      Month, Day (e.g. Oct 11 = 10, 11)

      Note: Implements full and correct leap year rules.

      Modification history: Created by Steve Morley (in July '05), 
      rewritten for Python in October 2009

\setlength{\parskip}{1ex}
    \end{boxedminipage}

    \label{spacepy:utils:t_overlap}
    \index{spacepy \textit{(package)}!spacepy.utils \textit{(module)}!spacepy.utils.t\_overlap \textit{(function)}}

    \vspace{0.5ex}

\hspace{.8\funcindent}\begin{boxedminipage}{\funcwidth}

    \raggedright \textbf{t\_overlap}(\textit{ts1}, \textit{ts2})

    \vspace{-1.5ex}

    \rule{\textwidth}{0.5\fboxrule}
\setlength{\parskip}{2ex}
    Finds the overlapping elements in two lists of datetime objects

    (section) Returns:

      \begin{itemize}
      \setlength{\parskip}{0.6ex}
        \item indices of 1 within interval of 2, \& vice versa

      \end{itemize}

    (section) Example:

      \begin{itemize}
      \setlength{\parskip}{0.6ex}
        \item Given two series of datetime objects, event\_dates and 
          omni['Time']:

      \end{itemize}

\begin{alltt}
\pysrcprompt{{\textgreater}{\textgreater}{\textgreater} }\pysrckeyword{import} spacepy.utils \pysrckeyword{as} utils
\pysrcprompt{{\textgreater}{\textgreater}{\textgreater} }[einds,oinds] = utils.t\_overlap(event\_dates, omni[\pysrcstring{'Time'}])
\pysrcprompt{{\textgreater}{\textgreater}{\textgreater} }omni\_time = omni[\pysrcstring{'Time'}][oinds[0]:oinds[-1]+1]
\pysrcprompt{{\textgreater}{\textgreater}{\textgreater} }\pysrckeyword{print} omni\_time
\pysrcoutput{[datetime.datetime(2007, 5, 5, 17, 57, 30), datetime.datetime(2007, 5, 5, 18, 2, 30),}
\pysrcoutput{... , datetime.datetime(2007, 5, 10, 4, 57, 30)]}\end{alltt}
\setlength{\parskip}{1ex}
    \end{boxedminipage}

    \label{spacepy:utils:t_common}
    \index{spacepy \textit{(package)}!spacepy.utils \textit{(module)}!spacepy.utils.t\_common \textit{(function)}}

    \vspace{0.5ex}

\hspace{.8\funcindent}\begin{boxedminipage}{\funcwidth}

    \raggedright \textbf{t\_common}(\textit{ts1}, \textit{ts2}, \textit{mask\_only}={\tt True})

    \vspace{-1.5ex}

    \rule{\textwidth}{0.5\fboxrule}
\setlength{\parskip}{2ex}
    Finds the elements in a list of datetime objects present in another

    (section) Returns:

      \begin{itemize}
      \setlength{\parskip}{0.6ex}
        \item Two element tuple of truth tables (of 1 present in 2, \& vice 
          versa)

      \end{itemize}

\setlength{\parskip}{1ex}
    \end{boxedminipage}

    \label{spacepy:utils:ShueMP}
    \index{spacepy \textit{(package)}!spacepy.utils \textit{(module)}!spacepy.utils.ShueMP \textit{(function)}}

    \vspace{0.5ex}

\hspace{.8\funcindent}\begin{boxedminipage}{\funcwidth}

    \raggedright \textbf{ShueMP}(\textit{P}, \textit{Bz})

    \vspace{-1.5ex}

    \rule{\textwidth}{0.5\fboxrule}
\setlength{\parskip}{2ex}
    Calculates the Shue et al. (1997) subsolar magnetopause radius

    Ported from Drew Turner's (LASP) MatLab script

    (section) Inputs:

      SW ram pressure [nPa], IMF Bz (GSM) [nT]

    (section) Output:

      Magnetopause (sub-solar point) standoff distance [Re]

\setlength{\parskip}{1ex}
    \end{boxedminipage}

    \label{spacepy:utils:MoldwinPP}
    \index{spacepy \textit{(package)}!spacepy.utils \textit{(module)}!spacepy.utils.MoldwinPP \textit{(function)}}

    \vspace{0.5ex}

\hspace{.8\funcindent}\begin{boxedminipage}{\funcwidth}

    \raggedright \textbf{MoldwinPP}(\textit{Kp})

    \vspace{-1.5ex}

    \rule{\textwidth}{0.5\fboxrule}
\setlength{\parskip}{2ex}
    Calculates the Moldwin et al. (2002) subsolar magnetopause radius

    Inputs: Kp index (uses max Kp from prev 12 hrs) Output: Plasmapause 
    radius [Re]

\setlength{\parskip}{1ex}
    \end{boxedminipage}

    \label{spacepy:utils:readOMNI}
    \index{spacepy \textit{(package)}!spacepy.utils \textit{(module)}!spacepy.utils.readOMNI \textit{(function)}}

    \vspace{0.5ex}

\hspace{.8\funcindent}\begin{boxedminipage}{\funcwidth}

    \raggedright \textbf{readOMNI}(\textit{fname})

    \vspace{-1.5ex}

    \rule{\textwidth}{0.5\fboxrule}
\setlength{\parskip}{2ex}
    Read amalgamated OMNI2 file into dictionary of data arrays

    Input: filename (str)

    Output: dictionary with datetime objects and data arrays

\setlength{\parskip}{1ex}
    \end{boxedminipage}

    \label{spacepy:utils:readOMNIhi}
    \index{spacepy \textit{(package)}!spacepy.utils \textit{(module)}!spacepy.utils.readOMNIhi \textit{(function)}}

    \vspace{0.5ex}

\hspace{.8\funcindent}\begin{boxedminipage}{\funcwidth}

    \raggedright \textbf{readOMNIhi}(\textit{fname})

    \vspace{-1.5ex}

    \rule{\textwidth}{0.5\fboxrule}
\setlength{\parskip}{2ex}
    Read high-resOMNI file into dictionary of data arrays

    Input: filename (str)

    Output: dictionary with datetime objects and data arrays

\setlength{\parskip}{1ex}
    \end{boxedminipage}

    \label{spacepy:utils:mu_ai}
    \index{spacepy \textit{(package)}!spacepy.utils \textit{(module)}!spacepy.utils.mu\_ai \textit{(function)}}

    \vspace{0.5ex}

\hspace{.8\funcindent}\begin{boxedminipage}{\funcwidth}

    \raggedright \textbf{mu\_ai}(\textit{energy}, \textit{b}={\tt 1e-07}, \textit{rme}={\tt 0.511})

    \vspace{-1.5ex}

    \rule{\textwidth}{0.5\fboxrule}
\setlength{\parskip}{2ex}
    Calculate 1st adiabatic invariant given energy in [MeV/G]

    Input: energy (req'd) in MeV

    Uses E = Ek + E0, E = sqrt(p{\textasciicircum}2c{\textasciicircum}2 + 
    E0{\textasciicircum}2); Then uses mu = p{\textasciicircum}2/2mB to 
    arrive at mu = (p{\textasciicircum}2c{\textasciicircum}2)/(2*E0*B)

\setlength{\parskip}{1ex}
    \end{boxedminipage}

    \label{spacepy:utils:dm_ll}
    \index{spacepy \textit{(package)}!spacepy.utils \textit{(module)}!spacepy.utils.dm\_ll \textit{(function)}}

    \vspace{0.5ex}

\hspace{.8\funcindent}\begin{boxedminipage}{\funcwidth}

    \raggedright \textbf{dm\_ll}(\textit{kp}={\tt 2.7}, \textit{l}={\tt 6.6})

    \vspace{-1.5ex}

    \rule{\textwidth}{0.5\fboxrule}
\setlength{\parskip}{2ex}
    Magnetic field diffusion coefficient from Brautigam and Albert

\setlength{\parskip}{1ex}
    \end{boxedminipage}

    \label{spacepy:utils:windowmean}
    \index{spacepy \textit{(package)}!spacepy.utils \textit{(module)}!spacepy.utils.windowmean \textit{(function)}}

    \vspace{0.5ex}

\hspace{.8\funcindent}\begin{boxedminipage}{\funcwidth}

    \raggedright \textbf{windowmean}(\textit{data}, \textit{time}={\tt \texttt{[}\texttt{]}}, \textit{winsize}={\tt 0}, \textit{overlap}={\tt 0}, \textit{pts}={\tt True})

    \vspace{-1.5ex}

    \rule{\textwidth}{0.5\fboxrule}
\setlength{\parskip}{2ex}
    Windowing mean function, window overlap is user defined

    Inputs: data - 1D series of points; time - series of timestamps, 
    optional (format as numeric or datetime); For non-overlapping windows 
    set overlap to zero. e.g.,

\begin{alltt}
\pysrcprompt{{\textgreater}{\textgreater}{\textgreater} }wsize, olap = datetime.timedelta(1), datetime.timedelta(0,3600)\end{alltt}
\begin{alltt}
\pysrcprompt{{\textgreater}{\textgreater}{\textgreater} }outtime, outdata = windowmean(data, time, winsize=wsize, overlap=olap)\end{alltt}
    where the time, winsize and overlap are either numberic or datetime 
    objects, in this example the window size is 1 day and the overlap is 1 
    hour.

    Caveats: This is a quick and dirty function - it is NOT optimised, at 
    all.

\setlength{\parskip}{1ex}
    \end{boxedminipage}

    \label{spacepy:utils:medabsdev}
    \index{spacepy \textit{(package)}!spacepy.utils \textit{(module)}!spacepy.utils.medabsdev \textit{(function)}}

    \vspace{0.5ex}

\hspace{.8\funcindent}\begin{boxedminipage}{\funcwidth}

    \raggedright \textbf{medabsdev}(\textit{series})

    \vspace{-1.5ex}

    \rule{\textwidth}{0.5\fboxrule}
\setlength{\parskip}{2ex}
    Calculate median absolute deviation of a given input series

    Median absolute deviation (MAD) is a robust and resistant measure of 
    the spread of a sample (same purpose as standard deviation). The MAD is
    preferred to the interquartile range as the interquartile range only 
    shows 50\% of the data whereas the MAD uses all data but remains robust
    and resistant. See e.g. Wilks, Statistical methods for the Atmospheric 
    Sciences, 1995, Ch. 3.

    This implementation is robust to presence of NaNs

    Example: Find the median absolute deviation of a data set. Here we use 
    the log- normal distribution fitted to the population of sawtooth 
    intervals, see Morley and Henderson, Comment, Geophysical Research 
    Letters, 2009.

\begin{alltt}
\pysrcprompt{{\textgreater}{\textgreater}{\textgreater} }data = numpy.random.lognormal(mean=5.1458, sigma=0.302313, size=30)
\pysrcprompt{{\textgreater}{\textgreater}{\textgreater} }\pysrckeyword{print} data
\pysrcoutput{array([ 181.28078923,  131.18152745, ... , 141.15455416, 160.88972791])}
\pysrcoutput{}\pysrcprompt{{\textgreater}{\textgreater}{\textgreater} }utils.medabsdev(data)
\pysrcoutput{28.346646721370192}\end{alltt}
\setlength{\parskip}{1ex}
    \end{boxedminipage}

    \label{spacepy:utils:makePoly}
    \index{spacepy \textit{(package)}!spacepy.utils \textit{(module)}!spacepy.utils.makePoly \textit{(function)}}

    \vspace{0.5ex}

\hspace{.8\funcindent}\begin{boxedminipage}{\funcwidth}

    \raggedright \textbf{makePoly}(\textit{x}, \textit{y1}, \textit{y2}, \textit{face}={\tt \texttt{'}\texttt{blue}\texttt{'}}, \textit{alpha}={\tt 0.5})

    \vspace{-1.5ex}

    \rule{\textwidth}{0.5\fboxrule}
\setlength{\parskip}{2ex}
    Make filled polygon for plotting

    Can be replaced by built-in matplotlib function fill\_between

\begin{alltt}
\pysrcprompt{{\textgreater}{\textgreater}{\textgreater} }poly0c = makePoly(x, ci\_low, ci\_high, face=\pysrcstring{'red'}, alpha=0.8)
\pysrcprompt{{\textgreater}{\textgreater}{\textgreater} }ax0.add\_patch(poly0qc)\end{alltt}
\setlength{\parskip}{1ex}
    \end{boxedminipage}

    \label{spacepy:utils:binHisto}
    \index{spacepy \textit{(package)}!spacepy.utils \textit{(module)}!spacepy.utils.binHisto \textit{(function)}}

    \vspace{0.5ex}

\hspace{.8\funcindent}\begin{boxedminipage}{\funcwidth}

    \raggedright \textbf{binHisto}(\textit{data})

    \vspace{-1.5ex}

    \rule{\textwidth}{0.5\fboxrule}
\setlength{\parskip}{2ex}
    Calculates bin width and number of bins for histogram using 
    Freedman-Diaconis rule

\setlength{\parskip}{1ex}
    \end{boxedminipage}


%%%%%%%%%%%%%%%%%%%%%%%%%%%%%%%%%%%%%%%%%%%%%%%%%%%%%%%%%%%%%%%%%%%%%%%%%%%
%%                               Variables                               %%
%%%%%%%%%%%%%%%%%%%%%%%%%%%%%%%%%%%%%%%%%%%%%%%%%%%%%%%%%%%%%%%%%%%%%%%%%%%

  \subsection{Variables}

    \vspace{-1cm}
\hspace{\varindent}\begin{longtable}{|p{\varnamewidth}|p{\vardescrwidth}|l}
\cline{1-2}
\cline{1-2} \centering \textbf{Name} & \centering \textbf{Description}& \\
\cline{1-2}
\endhead\cline{1-2}\multicolumn{3}{r}{\small\textit{continued on next page}}\\\endfoot\cline{1-2}
\endlastfoot\raggedright \_\-\_\-p\-a\-c\-k\-a\-g\-e\-\_\-\_\- & \raggedright \textbf{Value:} 
{\tt \texttt{'}\texttt{spacepy}\texttt{'}}&\\
\cline{1-2}
\end{longtable}

    \index{spacepy \textit{(package)}!spacepy.utils \textit{(module)}|)}
